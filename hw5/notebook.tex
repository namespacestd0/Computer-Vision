
% Default to the notebook output style

    


% Inherit from the specified cell style.




    
\documentclass[11pt]{article}

    
    
    \usepackage[T1]{fontenc}
    % Nicer default font (+ math font) than Computer Modern for most use cases
    \usepackage{mathpazo}

    % Basic figure setup, for now with no caption control since it's done
    % automatically by Pandoc (which extracts ![](path) syntax from Markdown).
    \usepackage{graphicx}
    % We will generate all images so they have a width \maxwidth. This means
    % that they will get their normal width if they fit onto the page, but
    % are scaled down if they would overflow the margins.
    \makeatletter
    \def\maxwidth{\ifdim\Gin@nat@width>\linewidth\linewidth
    \else\Gin@nat@width\fi}
    \makeatother
    \let\Oldincludegraphics\includegraphics
    % Set max figure width to be 80% of text width, for now hardcoded.
    \renewcommand{\includegraphics}[1]{\Oldincludegraphics[width=.8\maxwidth]{#1}}
    % Ensure that by default, figures have no caption (until we provide a
    % proper Figure object with a Caption API and a way to capture that
    % in the conversion process - todo).
    \usepackage{caption}
    \DeclareCaptionLabelFormat{nolabel}{}
    \captionsetup{labelformat=nolabel}

    \usepackage{adjustbox} % Used to constrain images to a maximum size 
    \usepackage{xcolor} % Allow colors to be defined
    \usepackage{enumerate} % Needed for markdown enumerations to work
    \usepackage{geometry} % Used to adjust the document margins
    \usepackage{amsmath} % Equations
    \usepackage{amssymb} % Equations
    \usepackage{textcomp} % defines textquotesingle
    % Hack from http://tex.stackexchange.com/a/47451/13684:
    \AtBeginDocument{%
        \def\PYZsq{\textquotesingle}% Upright quotes in Pygmentized code
    }
    \usepackage{upquote} % Upright quotes for verbatim code
    \usepackage{eurosym} % defines \euro
    \usepackage[mathletters]{ucs} % Extended unicode (utf-8) support
    \usepackage[utf8x]{inputenc} % Allow utf-8 characters in the tex document
    \usepackage{fancyvrb} % verbatim replacement that allows latex
    \usepackage{grffile} % extends the file name processing of package graphics 
                         % to support a larger range 
    % The hyperref package gives us a pdf with properly built
    % internal navigation ('pdf bookmarks' for the table of contents,
    % internal cross-reference links, web links for URLs, etc.)
    \usepackage{hyperref}
    \usepackage{longtable} % longtable support required by pandoc >1.10
    \usepackage{booktabs}  % table support for pandoc > 1.12.2
    \usepackage[inline]{enumitem} % IRkernel/repr support (it uses the enumerate* environment)
    \usepackage[normalem]{ulem} % ulem is needed to support strikethroughs (\sout)
                                % normalem makes italics be italics, not underlines
    

    
    
    % Colors for the hyperref package
    \definecolor{urlcolor}{rgb}{0,.145,.698}
    \definecolor{linkcolor}{rgb}{.71,0.21,0.01}
    \definecolor{citecolor}{rgb}{.12,.54,.11}

    % ANSI colors
    \definecolor{ansi-black}{HTML}{3E424D}
    \definecolor{ansi-black-intense}{HTML}{282C36}
    \definecolor{ansi-red}{HTML}{E75C58}
    \definecolor{ansi-red-intense}{HTML}{B22B31}
    \definecolor{ansi-green}{HTML}{00A250}
    \definecolor{ansi-green-intense}{HTML}{007427}
    \definecolor{ansi-yellow}{HTML}{DDB62B}
    \definecolor{ansi-yellow-intense}{HTML}{B27D12}
    \definecolor{ansi-blue}{HTML}{208FFB}
    \definecolor{ansi-blue-intense}{HTML}{0065CA}
    \definecolor{ansi-magenta}{HTML}{D160C4}
    \definecolor{ansi-magenta-intense}{HTML}{A03196}
    \definecolor{ansi-cyan}{HTML}{60C6C8}
    \definecolor{ansi-cyan-intense}{HTML}{258F8F}
    \definecolor{ansi-white}{HTML}{C5C1B4}
    \definecolor{ansi-white-intense}{HTML}{A1A6B2}

    % commands and environments needed by pandoc snippets
    % extracted from the output of `pandoc -s`
    \providecommand{\tightlist}{%
      \setlength{\itemsep}{0pt}\setlength{\parskip}{0pt}}
    \DefineVerbatimEnvironment{Highlighting}{Verbatim}{commandchars=\\\{\}}
    % Add ',fontsize=\small' for more characters per line
    \newenvironment{Shaded}{}{}
    \newcommand{\KeywordTok}[1]{\textcolor[rgb]{0.00,0.44,0.13}{\textbf{{#1}}}}
    \newcommand{\DataTypeTok}[1]{\textcolor[rgb]{0.56,0.13,0.00}{{#1}}}
    \newcommand{\DecValTok}[1]{\textcolor[rgb]{0.25,0.63,0.44}{{#1}}}
    \newcommand{\BaseNTok}[1]{\textcolor[rgb]{0.25,0.63,0.44}{{#1}}}
    \newcommand{\FloatTok}[1]{\textcolor[rgb]{0.25,0.63,0.44}{{#1}}}
    \newcommand{\CharTok}[1]{\textcolor[rgb]{0.25,0.44,0.63}{{#1}}}
    \newcommand{\StringTok}[1]{\textcolor[rgb]{0.25,0.44,0.63}{{#1}}}
    \newcommand{\CommentTok}[1]{\textcolor[rgb]{0.38,0.63,0.69}{\textit{{#1}}}}
    \newcommand{\OtherTok}[1]{\textcolor[rgb]{0.00,0.44,0.13}{{#1}}}
    \newcommand{\AlertTok}[1]{\textcolor[rgb]{1.00,0.00,0.00}{\textbf{{#1}}}}
    \newcommand{\FunctionTok}[1]{\textcolor[rgb]{0.02,0.16,0.49}{{#1}}}
    \newcommand{\RegionMarkerTok}[1]{{#1}}
    \newcommand{\ErrorTok}[1]{\textcolor[rgb]{1.00,0.00,0.00}{\textbf{{#1}}}}
    \newcommand{\NormalTok}[1]{{#1}}
    
    % Additional commands for more recent versions of Pandoc
    \newcommand{\ConstantTok}[1]{\textcolor[rgb]{0.53,0.00,0.00}{{#1}}}
    \newcommand{\SpecialCharTok}[1]{\textcolor[rgb]{0.25,0.44,0.63}{{#1}}}
    \newcommand{\VerbatimStringTok}[1]{\textcolor[rgb]{0.25,0.44,0.63}{{#1}}}
    \newcommand{\SpecialStringTok}[1]{\textcolor[rgb]{0.73,0.40,0.53}{{#1}}}
    \newcommand{\ImportTok}[1]{{#1}}
    \newcommand{\DocumentationTok}[1]{\textcolor[rgb]{0.73,0.13,0.13}{\textit{{#1}}}}
    \newcommand{\AnnotationTok}[1]{\textcolor[rgb]{0.38,0.63,0.69}{\textbf{\textit{{#1}}}}}
    \newcommand{\CommentVarTok}[1]{\textcolor[rgb]{0.38,0.63,0.69}{\textbf{\textit{{#1}}}}}
    \newcommand{\VariableTok}[1]{\textcolor[rgb]{0.10,0.09,0.49}{{#1}}}
    \newcommand{\ControlFlowTok}[1]{\textcolor[rgb]{0.00,0.44,0.13}{\textbf{{#1}}}}
    \newcommand{\OperatorTok}[1]{\textcolor[rgb]{0.40,0.40,0.40}{{#1}}}
    \newcommand{\BuiltInTok}[1]{{#1}}
    \newcommand{\ExtensionTok}[1]{{#1}}
    \newcommand{\PreprocessorTok}[1]{\textcolor[rgb]{0.74,0.48,0.00}{{#1}}}
    \newcommand{\AttributeTok}[1]{\textcolor[rgb]{0.49,0.56,0.16}{{#1}}}
    \newcommand{\InformationTok}[1]{\textcolor[rgb]{0.38,0.63,0.69}{\textbf{\textit{{#1}}}}}
    \newcommand{\WarningTok}[1]{\textcolor[rgb]{0.38,0.63,0.69}{\textbf{\textit{{#1}}}}}
    
    
    % Define a nice break command that doesn't care if a line doesn't already
    % exist.
    \def\br{\hspace*{\fill} \\* }
    % Math Jax compatability definitions
    \def\gt{>}
    \def\lt{<}
    % Document parameters
    \title{CSE152HW05}
    
    
    

    % Pygments definitions
    
\makeatletter
\def\PY@reset{\let\PY@it=\relax \let\PY@bf=\relax%
    \let\PY@ul=\relax \let\PY@tc=\relax%
    \let\PY@bc=\relax \let\PY@ff=\relax}
\def\PY@tok#1{\csname PY@tok@#1\endcsname}
\def\PY@toks#1+{\ifx\relax#1\empty\else%
    \PY@tok{#1}\expandafter\PY@toks\fi}
\def\PY@do#1{\PY@bc{\PY@tc{\PY@ul{%
    \PY@it{\PY@bf{\PY@ff{#1}}}}}}}
\def\PY#1#2{\PY@reset\PY@toks#1+\relax+\PY@do{#2}}

\expandafter\def\csname PY@tok@w\endcsname{\def\PY@tc##1{\textcolor[rgb]{0.73,0.73,0.73}{##1}}}
\expandafter\def\csname PY@tok@c\endcsname{\let\PY@it=\textit\def\PY@tc##1{\textcolor[rgb]{0.25,0.50,0.50}{##1}}}
\expandafter\def\csname PY@tok@cp\endcsname{\def\PY@tc##1{\textcolor[rgb]{0.74,0.48,0.00}{##1}}}
\expandafter\def\csname PY@tok@k\endcsname{\let\PY@bf=\textbf\def\PY@tc##1{\textcolor[rgb]{0.00,0.50,0.00}{##1}}}
\expandafter\def\csname PY@tok@kp\endcsname{\def\PY@tc##1{\textcolor[rgb]{0.00,0.50,0.00}{##1}}}
\expandafter\def\csname PY@tok@kt\endcsname{\def\PY@tc##1{\textcolor[rgb]{0.69,0.00,0.25}{##1}}}
\expandafter\def\csname PY@tok@o\endcsname{\def\PY@tc##1{\textcolor[rgb]{0.40,0.40,0.40}{##1}}}
\expandafter\def\csname PY@tok@ow\endcsname{\let\PY@bf=\textbf\def\PY@tc##1{\textcolor[rgb]{0.67,0.13,1.00}{##1}}}
\expandafter\def\csname PY@tok@nb\endcsname{\def\PY@tc##1{\textcolor[rgb]{0.00,0.50,0.00}{##1}}}
\expandafter\def\csname PY@tok@nf\endcsname{\def\PY@tc##1{\textcolor[rgb]{0.00,0.00,1.00}{##1}}}
\expandafter\def\csname PY@tok@nc\endcsname{\let\PY@bf=\textbf\def\PY@tc##1{\textcolor[rgb]{0.00,0.00,1.00}{##1}}}
\expandafter\def\csname PY@tok@nn\endcsname{\let\PY@bf=\textbf\def\PY@tc##1{\textcolor[rgb]{0.00,0.00,1.00}{##1}}}
\expandafter\def\csname PY@tok@ne\endcsname{\let\PY@bf=\textbf\def\PY@tc##1{\textcolor[rgb]{0.82,0.25,0.23}{##1}}}
\expandafter\def\csname PY@tok@nv\endcsname{\def\PY@tc##1{\textcolor[rgb]{0.10,0.09,0.49}{##1}}}
\expandafter\def\csname PY@tok@no\endcsname{\def\PY@tc##1{\textcolor[rgb]{0.53,0.00,0.00}{##1}}}
\expandafter\def\csname PY@tok@nl\endcsname{\def\PY@tc##1{\textcolor[rgb]{0.63,0.63,0.00}{##1}}}
\expandafter\def\csname PY@tok@ni\endcsname{\let\PY@bf=\textbf\def\PY@tc##1{\textcolor[rgb]{0.60,0.60,0.60}{##1}}}
\expandafter\def\csname PY@tok@na\endcsname{\def\PY@tc##1{\textcolor[rgb]{0.49,0.56,0.16}{##1}}}
\expandafter\def\csname PY@tok@nt\endcsname{\let\PY@bf=\textbf\def\PY@tc##1{\textcolor[rgb]{0.00,0.50,0.00}{##1}}}
\expandafter\def\csname PY@tok@nd\endcsname{\def\PY@tc##1{\textcolor[rgb]{0.67,0.13,1.00}{##1}}}
\expandafter\def\csname PY@tok@s\endcsname{\def\PY@tc##1{\textcolor[rgb]{0.73,0.13,0.13}{##1}}}
\expandafter\def\csname PY@tok@sd\endcsname{\let\PY@it=\textit\def\PY@tc##1{\textcolor[rgb]{0.73,0.13,0.13}{##1}}}
\expandafter\def\csname PY@tok@si\endcsname{\let\PY@bf=\textbf\def\PY@tc##1{\textcolor[rgb]{0.73,0.40,0.53}{##1}}}
\expandafter\def\csname PY@tok@se\endcsname{\let\PY@bf=\textbf\def\PY@tc##1{\textcolor[rgb]{0.73,0.40,0.13}{##1}}}
\expandafter\def\csname PY@tok@sr\endcsname{\def\PY@tc##1{\textcolor[rgb]{0.73,0.40,0.53}{##1}}}
\expandafter\def\csname PY@tok@ss\endcsname{\def\PY@tc##1{\textcolor[rgb]{0.10,0.09,0.49}{##1}}}
\expandafter\def\csname PY@tok@sx\endcsname{\def\PY@tc##1{\textcolor[rgb]{0.00,0.50,0.00}{##1}}}
\expandafter\def\csname PY@tok@m\endcsname{\def\PY@tc##1{\textcolor[rgb]{0.40,0.40,0.40}{##1}}}
\expandafter\def\csname PY@tok@gh\endcsname{\let\PY@bf=\textbf\def\PY@tc##1{\textcolor[rgb]{0.00,0.00,0.50}{##1}}}
\expandafter\def\csname PY@tok@gu\endcsname{\let\PY@bf=\textbf\def\PY@tc##1{\textcolor[rgb]{0.50,0.00,0.50}{##1}}}
\expandafter\def\csname PY@tok@gd\endcsname{\def\PY@tc##1{\textcolor[rgb]{0.63,0.00,0.00}{##1}}}
\expandafter\def\csname PY@tok@gi\endcsname{\def\PY@tc##1{\textcolor[rgb]{0.00,0.63,0.00}{##1}}}
\expandafter\def\csname PY@tok@gr\endcsname{\def\PY@tc##1{\textcolor[rgb]{1.00,0.00,0.00}{##1}}}
\expandafter\def\csname PY@tok@ge\endcsname{\let\PY@it=\textit}
\expandafter\def\csname PY@tok@gs\endcsname{\let\PY@bf=\textbf}
\expandafter\def\csname PY@tok@gp\endcsname{\let\PY@bf=\textbf\def\PY@tc##1{\textcolor[rgb]{0.00,0.00,0.50}{##1}}}
\expandafter\def\csname PY@tok@go\endcsname{\def\PY@tc##1{\textcolor[rgb]{0.53,0.53,0.53}{##1}}}
\expandafter\def\csname PY@tok@gt\endcsname{\def\PY@tc##1{\textcolor[rgb]{0.00,0.27,0.87}{##1}}}
\expandafter\def\csname PY@tok@err\endcsname{\def\PY@bc##1{\setlength{\fboxsep}{0pt}\fcolorbox[rgb]{1.00,0.00,0.00}{1,1,1}{\strut ##1}}}
\expandafter\def\csname PY@tok@kc\endcsname{\let\PY@bf=\textbf\def\PY@tc##1{\textcolor[rgb]{0.00,0.50,0.00}{##1}}}
\expandafter\def\csname PY@tok@kd\endcsname{\let\PY@bf=\textbf\def\PY@tc##1{\textcolor[rgb]{0.00,0.50,0.00}{##1}}}
\expandafter\def\csname PY@tok@kn\endcsname{\let\PY@bf=\textbf\def\PY@tc##1{\textcolor[rgb]{0.00,0.50,0.00}{##1}}}
\expandafter\def\csname PY@tok@kr\endcsname{\let\PY@bf=\textbf\def\PY@tc##1{\textcolor[rgb]{0.00,0.50,0.00}{##1}}}
\expandafter\def\csname PY@tok@bp\endcsname{\def\PY@tc##1{\textcolor[rgb]{0.00,0.50,0.00}{##1}}}
\expandafter\def\csname PY@tok@fm\endcsname{\def\PY@tc##1{\textcolor[rgb]{0.00,0.00,1.00}{##1}}}
\expandafter\def\csname PY@tok@vc\endcsname{\def\PY@tc##1{\textcolor[rgb]{0.10,0.09,0.49}{##1}}}
\expandafter\def\csname PY@tok@vg\endcsname{\def\PY@tc##1{\textcolor[rgb]{0.10,0.09,0.49}{##1}}}
\expandafter\def\csname PY@tok@vi\endcsname{\def\PY@tc##1{\textcolor[rgb]{0.10,0.09,0.49}{##1}}}
\expandafter\def\csname PY@tok@vm\endcsname{\def\PY@tc##1{\textcolor[rgb]{0.10,0.09,0.49}{##1}}}
\expandafter\def\csname PY@tok@sa\endcsname{\def\PY@tc##1{\textcolor[rgb]{0.73,0.13,0.13}{##1}}}
\expandafter\def\csname PY@tok@sb\endcsname{\def\PY@tc##1{\textcolor[rgb]{0.73,0.13,0.13}{##1}}}
\expandafter\def\csname PY@tok@sc\endcsname{\def\PY@tc##1{\textcolor[rgb]{0.73,0.13,0.13}{##1}}}
\expandafter\def\csname PY@tok@dl\endcsname{\def\PY@tc##1{\textcolor[rgb]{0.73,0.13,0.13}{##1}}}
\expandafter\def\csname PY@tok@s2\endcsname{\def\PY@tc##1{\textcolor[rgb]{0.73,0.13,0.13}{##1}}}
\expandafter\def\csname PY@tok@sh\endcsname{\def\PY@tc##1{\textcolor[rgb]{0.73,0.13,0.13}{##1}}}
\expandafter\def\csname PY@tok@s1\endcsname{\def\PY@tc##1{\textcolor[rgb]{0.73,0.13,0.13}{##1}}}
\expandafter\def\csname PY@tok@mb\endcsname{\def\PY@tc##1{\textcolor[rgb]{0.40,0.40,0.40}{##1}}}
\expandafter\def\csname PY@tok@mf\endcsname{\def\PY@tc##1{\textcolor[rgb]{0.40,0.40,0.40}{##1}}}
\expandafter\def\csname PY@tok@mh\endcsname{\def\PY@tc##1{\textcolor[rgb]{0.40,0.40,0.40}{##1}}}
\expandafter\def\csname PY@tok@mi\endcsname{\def\PY@tc##1{\textcolor[rgb]{0.40,0.40,0.40}{##1}}}
\expandafter\def\csname PY@tok@il\endcsname{\def\PY@tc##1{\textcolor[rgb]{0.40,0.40,0.40}{##1}}}
\expandafter\def\csname PY@tok@mo\endcsname{\def\PY@tc##1{\textcolor[rgb]{0.40,0.40,0.40}{##1}}}
\expandafter\def\csname PY@tok@ch\endcsname{\let\PY@it=\textit\def\PY@tc##1{\textcolor[rgb]{0.25,0.50,0.50}{##1}}}
\expandafter\def\csname PY@tok@cm\endcsname{\let\PY@it=\textit\def\PY@tc##1{\textcolor[rgb]{0.25,0.50,0.50}{##1}}}
\expandafter\def\csname PY@tok@cpf\endcsname{\let\PY@it=\textit\def\PY@tc##1{\textcolor[rgb]{0.25,0.50,0.50}{##1}}}
\expandafter\def\csname PY@tok@c1\endcsname{\let\PY@it=\textit\def\PY@tc##1{\textcolor[rgb]{0.25,0.50,0.50}{##1}}}
\expandafter\def\csname PY@tok@cs\endcsname{\let\PY@it=\textit\def\PY@tc##1{\textcolor[rgb]{0.25,0.50,0.50}{##1}}}

\def\PYZbs{\char`\\}
\def\PYZus{\char`\_}
\def\PYZob{\char`\{}
\def\PYZcb{\char`\}}
\def\PYZca{\char`\^}
\def\PYZam{\char`\&}
\def\PYZlt{\char`\<}
\def\PYZgt{\char`\>}
\def\PYZsh{\char`\#}
\def\PYZpc{\char`\%}
\def\PYZdl{\char`\$}
\def\PYZhy{\char`\-}
\def\PYZsq{\char`\'}
\def\PYZdq{\char`\"}
\def\PYZti{\char`\~}
% for compatibility with earlier versions
\def\PYZat{@}
\def\PYZlb{[}
\def\PYZrb{]}
\makeatother


    % Exact colors from NB
    \definecolor{incolor}{rgb}{0.0, 0.0, 0.5}
    \definecolor{outcolor}{rgb}{0.545, 0.0, 0.0}



    
    % Prevent overflowing lines due to hard-to-break entities
    \sloppy 
    % Setup hyperref package
    \hypersetup{
      breaklinks=true,  % so long urls are correctly broken across lines
      colorlinks=true,
      urlcolor=urlcolor,
      linkcolor=linkcolor,
      citecolor=citecolor,
      }
    % Slightly bigger margins than the latex defaults
    
    \geometry{verbose,tmargin=1in,bmargin=1in,lmargin=1in,rmargin=1in}
    
    

    \begin{document}
    
    
    \maketitle
    
    

    
    \section{CSE 152 : Introduction to Computer Vision, Spring 2018 --
Assignment
5}\label{cse-152-introduction-to-computer-vision-spring-2018-assignment-5}

    \subsection{Introduction}\label{introduction}

\subsubsection{Objective}\label{objective}

Here we will study object recognition techniques, in particular, we will
try to recognize faces. \#\#\# Dataset The face data that we will use is
derived from the Yale Face Database. The database consists of 5760
images of 10 individuals, each under 9 poses and 64 different lighting
conditions. They can be used for benchmarking different algorithms.

In this project, we will only use 640 images corresponding to a frontal
orientation of the face. The faces are divided into five different
subsets. Subset 0 consists of images where the light source direction is
almost frontal, so that almost all of the face is brightly illuminated.
From subset 1 to 4, the light source is progressively moved toward the
horizon, so that the effects of shadows increase and not all pixels are
illuminated. The faces in subset 0 will be used as training images, and
subsets 1 to 4 will be used as test images.

    \subsection{Environment Setup}\label{environment-setup}

util.py requires cv2 library. Run "conda install opencv" in Powershell
if your are in Windows or Terminal if you are in Mac/Linux.

    \begin{Verbatim}[commandchars=\\\{\}]
{\color{incolor}In [{\color{incolor}1}]:} \PY{k+kn}{import} \PY{n+nn}{os}\PY{o}{,} \PY{n+nn}{sys}
        \PY{k+kn}{import} \PY{n+nn}{numpy} \PY{k}{as} \PY{n+nn}{np}
        \PY{k+kn}{import} \PY{n+nn}{numpy}\PY{n+nn}{.}\PY{n+nn}{linalg} \PY{k}{as} \PY{n+nn}{nl}
        \PY{k+kn}{import} \PY{n+nn}{scipy}\PY{n+nn}{.}\PY{n+nn}{linalg} \PY{k}{as} \PY{n+nn}{sl}
        \PY{k+kn}{from} \PY{n+nn}{imageio} \PY{k}{import} \PY{n}{imread}
        \PY{k+kn}{import} \PY{n+nn}{matplotlib}\PY{n+nn}{.}\PY{n+nn}{pyplot} \PY{k}{as} \PY{n+nn}{plt}
        \PY{k+kn}{from} \PY{n+nn}{scipy}\PY{n+nn}{.}\PY{n+nn}{ndimage}\PY{n+nn}{.}\PY{n+nn}{filters} \PY{k}{import} \PY{n}{convolve}
        \PY{k+kn}{from} \PY{n+nn}{scipy}\PY{n+nn}{.}\PY{n+nn}{ndimage}\PY{n+nn}{.}\PY{n+nn}{filters} \PY{k}{import} \PY{n}{gaussian\PYZus{}filter}
        \PY{k+kn}{from} \PY{n+nn}{sklearn}\PY{n+nn}{.}\PY{n+nn}{decomposition} \PY{k}{import} \PY{n}{PCA}
        \PY{k+kn}{import} \PY{n+nn}{cv2} 
\end{Verbatim}


    \begin{Verbatim}[commandchars=\\\{\}]
{\color{incolor}In [{\color{incolor}2}]:} \PY{k}{def} \PY{n+nf}{load\PYZus{}subset}\PY{p}{(}\PY{n}{subsets}\PY{p}{,} \PY{n}{base\PYZus{}path}\PY{o}{=}\PY{l+s+s1}{\PYZsq{}}\PY{l+s+s1}{yaleBfaces}\PY{l+s+s1}{\PYZsq{}}\PY{p}{)}\PY{p}{:}
            \PY{n}{imgs} \PY{o}{=} \PY{p}{[}\PY{p}{]}
            \PY{n}{labels} \PY{o}{=} \PY{p}{[}\PY{p}{]}
            \PY{n}{sequence} \PY{o}{=} \PY{p}{[}\PY{p}{]}
            \PY{k}{for} \PY{n}{subset} \PY{o+ow}{in} \PY{n}{subsets}\PY{p}{:}
                \PY{n}{directory} \PY{o}{=} \PY{n}{os}\PY{o}{.}\PY{n}{path}\PY{o}{.}\PY{n}{join}\PY{p}{(}\PY{n}{base\PYZus{}path}\PY{p}{,} \PY{l+s+s2}{\PYZdq{}}\PY{l+s+s2}{subset}\PY{l+s+s2}{\PYZdq{}} \PY{o}{+} \PY{n+nb}{str}\PY{p}{(}\PY{n}{subset}\PY{p}{)}\PY{p}{)}
                \PY{n}{files} \PY{o}{=} \PY{n}{os}\PY{o}{.}\PY{n}{listdir}\PY{p}{(}\PY{n}{directory}\PY{p}{)}
                \PY{k}{for} \PY{n}{img} \PY{o+ow}{in} \PY{n}{files}\PY{p}{:}
                    \PY{n}{face} \PY{o}{=} \PY{n}{cv2}\PY{o}{.}\PY{n}{imread}\PY{p}{(}\PY{n}{os}\PY{o}{.}\PY{n}{path}\PY{o}{.}\PY{n}{join}\PY{p}{(}\PY{n}{directory}\PY{p}{,}\PY{n}{img}\PY{p}{)}\PY{p}{,} \PY{n}{cv2}\PY{o}{.}\PY{n}{IMREAD\PYZus{}GRAYSCALE}\PY{p}{)}
                    \PY{n}{imgs}\PY{o}{.}\PY{n}{append}\PY{p}{(}\PY{n}{face}\PY{p}{)}
                    \PY{n}{labels}\PY{o}{.}\PY{n}{append}\PY{p}{(}\PY{n+nb}{int}\PY{p}{(}\PY{n}{img}\PY{o}{.}\PY{n}{split}\PY{p}{(}\PY{l+s+s1}{\PYZsq{}}\PY{l+s+s1}{person}\PY{l+s+s1}{\PYZsq{}}\PY{p}{)}\PY{p}{[}\PY{l+m+mi}{1}\PY{p}{]}\PY{o}{.}\PY{n}{split}\PY{p}{(}\PY{l+s+s1}{\PYZsq{}}\PY{l+s+s1}{\PYZus{}}\PY{l+s+s1}{\PYZsq{}}\PY{p}{)}\PY{p}{[}\PY{l+m+mi}{0}\PY{p}{]}\PY{p}{)}\PY{p}{)}
                    \PY{n}{sequence}\PY{o}{.}\PY{n}{append}\PY{p}{(}\PY{n+nb}{int}\PY{p}{(}\PY{n}{img}\PY{o}{.}\PY{n}{split}\PY{p}{(}\PY{l+s+s1}{\PYZsq{}}\PY{l+s+s1}{person}\PY{l+s+s1}{\PYZsq{}}\PY{p}{)}\PY{p}{[}\PY{l+m+mi}{1}\PY{p}{]}\PY{o}{.}\PY{n}{split}\PY{p}{(}\PY{l+s+s1}{\PYZsq{}}\PY{l+s+s1}{\PYZus{}}\PY{l+s+s1}{\PYZsq{}}\PY{p}{)}\PY{p}{[}\PY{l+m+mi}{1}\PY{p}{]}\PY{o}{.}\PY{n}{split}\PY{p}{(}\PY{l+s+s1}{\PYZsq{}}\PY{l+s+s1}{.}\PY{l+s+s1}{\PYZsq{}}\PY{p}{)}\PY{p}{[}\PY{l+m+mi}{0}\PY{p}{]}\PY{p}{)}\PY{p}{)}
            \PY{n}{imgs} \PY{o}{=} \PY{n}{np}\PY{o}{.}\PY{n}{array}\PY{p}{(}\PY{n}{imgs}\PY{p}{)}\PY{o}{.}\PY{n}{astype}\PY{p}{(}\PY{l+s+s2}{\PYZdq{}}\PY{l+s+s2}{float64}\PY{l+s+s2}{\PYZdq{}}\PY{p}{)}
            \PY{k}{return} \PY{n}{imgs}\PY{p}{,} \PY{n}{labels}\PY{p}{,} \PY{n}{sequence}
\end{Verbatim}


    \begin{Verbatim}[commandchars=\\\{\}]
{\color{incolor}In [{\color{incolor}3}]:} \PY{k}{def} \PY{n+nf}{draw\PYZus{}faces}\PY{p}{(}\PY{n}{img\PYZus{}list}\PY{p}{,} \PY{n}{col}\PY{o}{=}\PY{l+m+mi}{10}\PY{p}{)}\PY{p}{:}
            \PY{k}{if} \PY{n+nb}{len}\PY{p}{(}\PY{n}{img\PYZus{}list}\PY{p}{)} \PY{o}{\PYZlt{}} \PY{n}{col}\PY{p}{:}
                \PY{n}{col} \PY{o}{=} \PY{n+nb}{len}\PY{p}{(}\PY{n}{img\PYZus{}list}\PY{p}{)}
                \PY{n}{row} \PY{o}{=} \PY{l+m+mi}{1}
            \PY{k}{else}\PY{p}{:}
                \PY{n}{row} \PY{o}{=} \PY{n+nb}{int}\PY{p}{(}\PY{n+nb}{len}\PY{p}{(}\PY{n}{img\PYZus{}list}\PY{p}{)}\PY{o}{/}\PY{n}{col}\PY{p}{)}
            \PY{n}{fig} \PY{o}{=} \PY{n}{plt}\PY{o}{.}\PY{n}{figure}\PY{p}{(}\PY{n}{figsize} \PY{o}{=} \PY{p}{(}\PY{l+m+mi}{15}\PY{p}{,}\PY{l+m+mi}{2}\PY{o}{+}\PY{l+m+mf}{1.25}\PY{o}{*}\PY{n}{row}\PY{p}{)}\PY{p}{)}
            \PY{k}{for} \PY{n}{sub\PYZus{}img} \PY{o+ow}{in} \PY{n+nb}{range}\PY{p}{(}\PY{l+m+mi}{1}\PY{p}{,}\PY{n}{row}\PY{o}{*}\PY{n}{col}\PY{o}{+}\PY{l+m+mi}{1}\PY{p}{)}\PY{p}{:}
                \PY{n}{ax} \PY{o}{=} \PY{n}{fig}\PY{o}{.}\PY{n}{add\PYZus{}subplot}\PY{p}{(}\PY{n}{row}\PY{p}{,} \PY{n}{col}\PY{p}{,} \PY{n}{sub\PYZus{}img}\PY{p}{)}
                \PY{n}{ax}\PY{o}{.}\PY{n}{imshow}\PY{p}{(}\PY{n}{img\PYZus{}list}\PY{p}{[}\PY{n}{sub\PYZus{}img}\PY{o}{\PYZhy{}}\PY{l+m+mi}{1}\PY{p}{]}\PY{p}{,} \PY{n}{cmap}\PY{o}{=}\PY{l+s+s1}{\PYZsq{}}\PY{l+s+s1}{gray}\PY{l+s+s1}{\PYZsq{}}\PY{p}{)}
                \PY{n}{ax}\PY{o}{.}\PY{n}{axis}\PY{p}{(}\PY{l+s+s1}{\PYZsq{}}\PY{l+s+s1}{off}\PY{l+s+s1}{\PYZsq{}}\PY{p}{)}
            \PY{n}{plt}\PY{o}{.}\PY{n}{show}\PY{p}{(}\PY{p}{)}
\end{Verbatim}


    \subsection{Problem 1 (Programing): Naive Recognition (15
points)}\label{problem-1-programing-naive-recognition-15-points}

Perform face recognition by comparing the raw pixel values of the
images.

    \begin{Verbatim}[commandchars=\\\{\}]
{\color{incolor}In [{\color{incolor}4}]:} \PY{c+c1}{\PYZsh{} Let subset 0 be the train set}
        \PY{n}{img}\PY{p}{,} \PY{n}{label}\PY{p}{,} \PY{n}{seq} \PY{o}{=} \PY{n}{load\PYZus{}subset}\PY{p}{(}\PY{p}{[}\PY{l+m+mi}{0}\PY{p}{]}\PY{p}{)}
        
        \PY{k}{def} \PY{n+nf}{nn1\PYZus{}score}\PY{p}{(}\PY{n}{im}\PY{p}{,} \PY{n}{imgs} \PY{o}{=} \PY{n}{img}\PY{p}{,} \PY{n}{order} \PY{o}{=} \PY{k+kc}{None}\PY{p}{)}\PY{p}{:}
            \PY{n}{d} \PY{o}{=} \PY{n}{np}\PY{o}{.}\PY{n}{empty}\PY{p}{(}\PY{n+nb}{len}\PY{p}{(}\PY{n}{imgs}\PY{p}{)}\PY{p}{)}
            \PY{k}{for} \PY{n}{k} \PY{o+ow}{in} \PY{n+nb}{range}\PY{p}{(}\PY{n+nb}{len}\PY{p}{(}\PY{n}{imgs}\PY{p}{)}\PY{p}{)}\PY{p}{:}
                \PY{n}{d}\PY{p}{[}\PY{n}{k}\PY{p}{]} \PY{o}{=} \PY{n}{np}\PY{o}{.}\PY{n}{linalg}\PY{o}{.}\PY{n}{norm}\PY{p}{(}\PY{n}{imgs}\PY{p}{[}\PY{n}{k}\PY{p}{]}\PY{o}{.}\PY{n}{ravel}\PY{p}{(}\PY{p}{)} \PY{o}{\PYZhy{}} \PY{n}{im}\PY{o}{.}\PY{n}{ravel}\PY{p}{(}\PY{p}{)}\PY{p}{,} \PY{n}{order}\PY{p}{)}
            \PY{k}{return} \PY{n}{d}
        
        \PY{c+c1}{\PYZsh{} find nearest neighboor of im in set img}
        \PY{k}{def} \PY{n+nf}{nn1}\PY{p}{(}\PY{n}{im}\PY{p}{,} \PY{n}{imgs} \PY{o}{=} \PY{n}{img}\PY{p}{,} \PY{n}{order}\PY{o}{=}\PY{k+kc}{None}\PY{p}{)}\PY{p}{:}
            \PY{n}{d} \PY{o}{=} \PY{n}{nn1\PYZus{}score}\PY{p}{(}\PY{n}{im}\PY{p}{,} \PY{n}{imgs}\PY{p}{,} \PY{n}{order}\PY{p}{)}
            \PY{k}{return} \PY{n}{np}\PY{o}{.}\PY{n}{argmin}\PY{p}{(}\PY{n}{d}\PY{p}{)}
        
        \PY{k}{def} \PY{n+nf}{find\PYZus{}training\PYZus{}ind}\PY{p}{(}\PY{n}{index}\PY{p}{)}\PY{p}{:}
            \PY{k}{return} \PY{n}{np}\PY{o}{.}\PY{n}{where}\PY{p}{(}\PY{n}{np}\PY{o}{.}\PY{n}{array}\PY{p}{(}\PY{n}{label}\PY{p}{)}\PY{o}{==}\PY{n}{index}\PY{p}{)}
        
        \PY{c+c1}{\PYZsh{} find all photos in set img that belongs to person/class \PYZsh{}index}
        \PY{k}{def} \PY{n+nf}{find\PYZus{}training\PYZus{}set}\PY{p}{(}\PY{n}{index}\PY{p}{,} \PY{n}{img} \PY{o}{=} \PY{n}{img}\PY{p}{)}\PY{p}{:}
            \PY{n}{ind} \PY{o}{=} \PY{n}{find\PYZus{}training\PYZus{}ind}\PY{p}{(}\PY{n}{index}\PY{p}{)}
            \PY{k}{return} \PY{n}{np}\PY{o}{.}\PY{n}{array}\PY{p}{(}\PY{n}{img}\PY{p}{)}\PY{p}{[}\PY{n}{ind}\PY{p}{]}\PY{o}{.}\PY{n}{tolist}\PY{p}{(}\PY{p}{)}
        
        \PY{c+c1}{\PYZsh{} draw all photos in set img that belongs to person/class \PYZsh{}index}
        \PY{k}{def} \PY{n+nf}{draw\PYZus{}training\PYZus{}set}\PY{p}{(}\PY{n}{index}\PY{p}{,} \PY{n}{img} \PY{o}{=} \PY{k+kc}{None}\PY{p}{)}\PY{p}{:}
            \PY{n}{draw\PYZus{}faces}\PY{p}{(}\PY{n}{find\PYZus{}training\PYZus{}set}\PY{p}{(}\PY{n}{index}\PY{p}{)}\PY{p}{)}
            \PY{n+nb}{print}\PY{p}{(}\PY{l+s+s2}{\PYZdq{}}\PY{l+s+s2}{Class}\PY{l+s+s2}{\PYZdq{}}\PY{p}{,}\PY{n}{index}\PY{p}{,}\PY{l+s+s2}{\PYZdq{}}\PY{l+s+s2}{Training Set}\PY{l+s+s2}{\PYZdq{}}\PY{p}{)}
            \PY{k}{if} \PY{n}{img} \PY{o+ow}{is} \PY{o+ow}{not} \PY{k+kc}{None}\PY{p}{:}
                \PY{n}{d} \PY{o}{=} \PY{n}{nn1\PYZus{}score}\PY{p}{(}\PY{n}{img}\PY{p}{)}
                \PY{n}{d\PYZus{}of\PYZus{}interest} \PY{o}{=} \PY{n}{d}\PY{p}{[}\PY{n}{find\PYZus{}training\PYZus{}ind}\PY{p}{(}\PY{n}{index}\PY{p}{)}\PY{p}{]}
                \PY{n+nb}{print}\PY{p}{(}\PY{n}{d\PYZus{}of\PYZus{}interest}\PY{o}{.}\PY{n}{astype}\PY{p}{(}\PY{n+nb}{int}\PY{p}{)}\PY{p}{)}
                \PY{n+nb}{print}\PY{p}{(}\PY{n+nb}{int}\PY{p}{(}\PY{n}{d\PYZus{}of\PYZus{}interest}\PY{o}{.}\PY{n}{mean}\PY{p}{(}\PY{p}{)}\PY{p}{)}\PY{p}{,} \PY{n+nb}{int}\PY{p}{(}\PY{n}{np}\PY{o}{.}\PY{n}{median}\PY{p}{(}\PY{n}{d\PYZus{}of\PYZus{}interest}\PY{p}{)}\PY{p}{)}\PY{p}{)}
\end{Verbatim}


    \subsubsection{Use the nearest neighbor classifier (1-NN) using the l2
norm (i.e. Euclidean
distance).}\label{use-the-nearest-neighbor-classifier-1-nn-using-the-l2-norm-i.e.-euclidean-distance.}

\paragraph{1. (5 points) Accuracy}\label{points-accuracy}

    \begin{Verbatim}[commandchars=\\\{\}]
{\color{incolor}In [{\color{incolor}5}]:} \PY{c+c1}{\PYZsh{} classified all images of set \PYZsh{}i with nn1 algorithm}
        \PY{k}{def} \PY{n+nf}{predict\PYZus{}naive}\PY{p}{(}\PY{n}{test\PYZus{}imgs}\PY{p}{,} \PY{n}{train\PYZus{}imgs}\PY{p}{,} \PY{n}{train\PYZus{}labels}\PY{p}{,} \PY{n}{order} \PY{o}{=} \PY{k+kc}{None}\PY{p}{)}\PY{p}{:}
            \PY{n}{labels} \PY{o}{=} \PY{n}{np}\PY{o}{.}\PY{n}{empty}\PY{p}{(}\PY{n+nb}{len}\PY{p}{(}\PY{n}{test\PYZus{}imgs}\PY{p}{)}\PY{p}{)}
            \PY{k}{for} \PY{n}{j} \PY{o+ow}{in} \PY{n+nb}{range}\PY{p}{(}\PY{n+nb}{len}\PY{p}{(}\PY{n}{test\PYZus{}imgs}\PY{p}{)}\PY{p}{)}\PY{p}{:}
                    \PY{n}{labels}\PY{p}{[}\PY{n}{j}\PY{p}{]} \PY{o}{=} \PY{n}{train\PYZus{}labels}\PY{p}{[}\PY{n}{nn1}\PY{p}{(}\PY{n}{test\PYZus{}imgs}\PY{p}{[}\PY{n}{j}\PY{p}{]}\PY{p}{,}\PY{n}{train\PYZus{}imgs}\PY{p}{,}\PY{n}{order}\PY{p}{)}\PY{p}{]}
            \PY{k}{return} \PY{n}{labels}
        
        \PY{c+c1}{\PYZsh{} compare labels of recognition \PYZdq{}labels\PYZdq{} with reference in set \PYZsh{}i}
        \PY{k}{def} \PY{n+nf}{evaluate}\PY{p}{(}\PY{n}{labels}\PY{p}{,} \PY{n}{l}\PY{p}{,} \PY{n}{i}\PY{o}{=}\PY{l+s+s2}{\PYZdq{}}\PY{l+s+s2}{\PYZdq{}}\PY{p}{)}\PY{p}{:} \PY{c+c1}{\PYZsh{} l for ref\PYZhy{}labels}
            \PY{n+nb}{print}\PY{p}{(}\PY{l+s+s2}{\PYZdq{}}\PY{l+s+s2}{Subset}\PY{l+s+s2}{\PYZdq{}}\PY{p}{,} \PY{n}{i}\PY{p}{,} \PY{l+s+s2}{\PYZdq{}}\PY{l+s+s2}{accuracy: }\PY{l+s+s2}{\PYZdq{}} 
                  \PY{o}{+} \PY{n+nb}{str}\PY{p}{(}\PY{l+s+s2}{\PYZdq{}}\PY{l+s+si}{\PYZob{}:.6\PYZcb{}}\PY{l+s+s2}{\PYZdq{}}\PY{o}{.}\PY{n}{format}\PY{p}{(}\PY{n}{np}\PY{o}{.}\PY{n}{count\PYZus{}nonzero}\PY{p}{(}\PY{p}{(}\PY{n}{np}\PY{o}{.}\PY{n}{array}\PY{p}{(}\PY{n}{labels}\PY{p}{)}\PY{o}{\PYZhy{}}\PY{n}{l}\PY{p}{)} \PY{o}{==} \PY{l+m+mi}{0}\PY{p}{)} \PY{o}{/} \PY{n+nb}{len}\PY{p}{(}\PY{n}{labels}\PY{p}{)} \PY{o}{*} \PY{l+m+mi}{100}\PY{p}{)} \PY{o}{+} \PY{l+s+s2}{\PYZdq{}}\PY{l+s+s2}{\PYZpc{}}\PY{l+s+s2}{\PYZdq{}}\PY{p}{)}\PY{p}{)}
        
        \PY{c+c1}{\PYZsh{} report classification accuracy on test sets 1 to 4. }
        \PY{k}{def} \PY{n+nf}{report\PYZus{}1nn}\PY{p}{(}\PY{p}{)}\PY{p}{:}
            \PY{k}{for} \PY{n}{i} \PY{o+ow}{in} \PY{n+nb}{range}\PY{p}{(}\PY{l+m+mi}{1}\PY{p}{,}\PY{l+m+mi}{5}\PY{p}{)}\PY{p}{:} 
                \PY{n}{im}\PY{p}{,} \PY{n}{lb}\PY{p}{,} \PY{n}{\PYZus{}} \PY{o}{=} \PY{n}{load\PYZus{}subset}\PY{p}{(}\PY{p}{[}\PY{n}{i}\PY{p}{]}\PY{p}{)}
                \PY{n}{labels} \PY{o}{=} \PY{n}{predict\PYZus{}naive}\PY{p}{(}\PY{n}{im}\PY{p}{,}\PY{n}{img}\PY{p}{,}\PY{n}{label}\PY{p}{)}
                \PY{n}{evaluate}\PY{p}{(}\PY{n}{labels}\PY{p}{,} \PY{n}{lb}\PY{p}{,} \PY{n}{i}\PY{p}{)}
                
        \PY{n}{report\PYZus{}1nn}\PY{p}{(}\PY{p}{)}
\end{Verbatim}


    \begin{Verbatim}[commandchars=\\\{\}]
Subset 1 accuracy: 94.1667\%
Subset 2 accuracy: 51.6667\%
Subset 3 accuracy: 19.2857\%
Subset 4 accuracy: 15.2632\%

    \end{Verbatim}

    \paragraph{2. (5 points) Performance}\label{points-performance}

The results seem to make sense becuase firstly it shows that for a set
with larger lighting differences comparing to the training set the
recoginition rate is smaller than one with more similar lighting. So
that's why set 1 performs better than 2 and 2 better than 3 and so on.
And secondly the overall rate is not very good due to the limitation of
the algorithm. The computation is fairly quick though, which also aligns
with the nature of the algorithm.

\paragraph{3. (5 points) Misclassification
Examination}\label{points-misclassification-examination}

    \begin{Verbatim}[commandchars=\\\{\}]
{\color{incolor}In [{\color{incolor}6}]:} \PY{c+c1}{\PYZsh{} take photo set 2 here for example}
        \PY{n}{im\PYZus{}set2}\PY{p}{,} \PY{n}{lb\PYZus{}set2}\PY{p}{,} \PY{n}{\PYZus{}} \PY{o}{=} \PY{n}{load\PYZus{}subset}\PY{p}{(}\PY{p}{[}\PY{l+m+mi}{2}\PY{p}{]}\PY{p}{)}
        
        \PY{k}{def} \PY{n+nf}{demonstrate\PYZus{}classification}\PY{p}{(}\PY{n}{index}\PY{p}{,} \PY{n}{im} \PY{o}{=} \PY{n}{im\PYZus{}set2}\PY{p}{)}\PY{p}{:}
            \PY{n}{plt}\PY{o}{.}\PY{n}{gcf}\PY{p}{(}\PY{p}{)}\PY{o}{.}\PY{n}{set\PYZus{}size\PYZus{}inches}\PY{p}{(}\PY{l+m+mi}{5}\PY{p}{,}\PY{l+m+mi}{2}\PY{p}{)}
            
            \PY{n}{plt}\PY{o}{.}\PY{n}{subplot}\PY{p}{(}\PY{l+m+mi}{121}\PY{p}{)}
            \PY{n}{plt}\PY{o}{.}\PY{n}{imshow}\PY{p}{(}\PY{n}{im}\PY{p}{[}\PY{n}{index}\PY{p}{]}\PY{p}{,}\PY{n}{cmap}\PY{o}{=}\PY{l+s+s1}{\PYZsq{}}\PY{l+s+s1}{gray}\PY{l+s+s1}{\PYZsq{}}\PY{p}{)}
            \PY{n}{plt}\PY{o}{.}\PY{n}{title}\PY{p}{(}\PY{l+s+s2}{\PYZdq{}}\PY{l+s+s2}{Class }\PY{l+s+s2}{\PYZdq{}} \PY{o}{+} \PY{n+nb}{str}\PY{p}{(}\PY{n}{lb\PYZus{}set2}\PY{p}{[}\PY{n}{index}\PY{p}{]}\PY{p}{)} \PY{o}{+} \PY{l+s+s2}{\PYZdq{}}\PY{l+s+s2}{ Photo Sample}\PY{l+s+s2}{\PYZdq{}}\PY{p}{)}
            \PY{n}{plt}\PY{o}{.}\PY{n}{gca}\PY{p}{(}\PY{p}{)}\PY{o}{.}\PY{n}{axis}\PY{p}{(}\PY{l+s+s1}{\PYZsq{}}\PY{l+s+s1}{off}\PY{l+s+s1}{\PYZsq{}}\PY{p}{)}
        
            \PY{n}{plt}\PY{o}{.}\PY{n}{subplot}\PY{p}{(}\PY{l+m+mi}{122}\PY{p}{)}
            \PY{n}{result\PYZus{}index} \PY{o}{=} \PY{n}{nn1}\PY{p}{(}\PY{n}{im}\PY{p}{[}\PY{n}{index}\PY{p}{]}\PY{p}{)}
            \PY{n}{plt}\PY{o}{.}\PY{n}{imshow}\PY{p}{(}\PY{n}{img}\PY{p}{[}\PY{n}{result\PYZus{}index}\PY{p}{]}\PY{p}{,}\PY{n}{cmap}\PY{o}{=}\PY{l+s+s1}{\PYZsq{}}\PY{l+s+s1}{gray}\PY{l+s+s1}{\PYZsq{}}\PY{p}{)}
            \PY{n}{plt}\PY{o}{.}\PY{n}{title}\PY{p}{(}\PY{l+s+s2}{\PYZdq{}}\PY{l+s+s2}{Class }\PY{l+s+s2}{\PYZdq{}} \PY{o}{+} \PY{n+nb}{str}\PY{p}{(}\PY{n}{label}\PY{p}{[}\PY{n}{result\PYZus{}index}\PY{p}{]}\PY{p}{)} \PY{o}{+} \PY{l+s+s2}{\PYZdq{}}\PY{l+s+s2}{ Photo Matched}\PY{l+s+s2}{\PYZdq{}}\PY{p}{)}
            \PY{n}{plt}\PY{o}{.}\PY{n}{gca}\PY{p}{(}\PY{p}{)}\PY{o}{.}\PY{n}{axis}\PY{p}{(}\PY{l+s+s1}{\PYZsq{}}\PY{l+s+s1}{off}\PY{l+s+s1}{\PYZsq{}}\PY{p}{)}
        
            \PY{n}{plt}\PY{o}{.}\PY{n}{show}\PY{p}{(}\PY{p}{)}
            \PY{k}{return} \PY{n}{lb\PYZus{}set2}\PY{p}{[}\PY{n}{index}\PY{p}{]}\PY{p}{,} \PY{n}{label}\PY{p}{[}\PY{n}{result\PYZus{}index}\PY{p}{]}
        
        \PY{c+c1}{\PYZsh{} choose this misclassified instance}
        \PY{n}{image\PYZus{}choice} \PY{o}{=} \PY{l+m+mi}{1}
        \PY{n}{actual}\PY{p}{,} \PY{n}{recog} \PY{o}{=} \PY{n}{demonstrate\PYZus{}classification}\PY{p}{(}\PY{n}{image\PYZus{}choice}\PY{p}{)}
\end{Verbatim}


    \begin{center}
    \adjustimage{max size={0.9\linewidth}{0.9\paperheight}}{output_11_0.png}
    \end{center}
    { \hspace*{\fill} \\}
    
    The result above shows a class 8 photo has been wrongly classified to
class 4. Now examine the training sets for both class 8 and 4. And show
the 1-NN distances. It shows the shortest distance comes from the first
picture in class 4 instead of class 8 although the median distance is
shorter for class 8, the original set. The average shows otherwise. This
shows how the extreme values may interfere with the recognition process.

    \begin{Verbatim}[commandchars=\\\{\}]
{\color{incolor}In [{\color{incolor}7}]:} \PY{n}{draw\PYZus{}training\PYZus{}set}\PY{p}{(}\PY{n}{actual}\PY{p}{,}\PY{n}{im\PYZus{}set2}\PY{p}{[}\PY{n}{image\PYZus{}choice}\PY{p}{]}\PY{p}{)}
\end{Verbatim}


    \begin{center}
    \adjustimage{max size={0.9\linewidth}{0.9\paperheight}}{output_13_0.png}
    \end{center}
    { \hspace*{\fill} \\}
    
    \begin{Verbatim}[commandchars=\\\{\}]
Class 8 Training Set
[1814 2108 2012 2353 1797 2113 2375]
2082 2108

    \end{Verbatim}

    \begin{Verbatim}[commandchars=\\\{\}]
{\color{incolor}In [{\color{incolor}8}]:} \PY{n}{draw\PYZus{}training\PYZus{}set}\PY{p}{(}\PY{n}{recog}\PY{p}{,}\PY{n}{im\PYZus{}set2}\PY{p}{[}\PY{n}{image\PYZus{}choice}\PY{p}{]}\PY{p}{)}
\end{Verbatim}


    \begin{center}
    \adjustimage{max size={0.9\linewidth}{0.9\paperheight}}{output_14_0.png}
    \end{center}
    { \hspace*{\fill} \\}
    
    \begin{Verbatim}[commandchars=\\\{\}]
Class 4 Training Set
[1684 2145 1814 2089 1814 2078 2194]
1974 2078

    \end{Verbatim}

    Take another recognition result in the same set to study. This time,
photos from the wrongly recognized group show overall better distance
result, in both cloest neighbor, average distance, and median distance
comparing with the training sets. This indicates this algorithm does not
effectively take consideration of the distinguishing features of objects
in comparision, and is largely affected by other elements like maybe
alignment, brightness(affected by \textbf{direction of light}), etc.

    \begin{Verbatim}[commandchars=\\\{\}]
{\color{incolor}In [{\color{incolor}9}]:} \PY{n}{image\PYZus{}choice} \PY{o}{=} \PY{l+m+mi}{7}
        \PY{n}{actual}\PY{p}{,} \PY{n}{recog} \PY{o}{=} \PY{n}{demonstrate\PYZus{}classification}\PY{p}{(}\PY{n}{image\PYZus{}choice}\PY{p}{)}
        \PY{c+c1}{\PYZsh{} class 4 contents see above section}
\end{Verbatim}


    \begin{center}
    \adjustimage{max size={0.9\linewidth}{0.9\paperheight}}{output_16_0.png}
    \end{center}
    { \hspace*{\fill} \\}
    
    \begin{Verbatim}[commandchars=\\\{\}]
{\color{incolor}In [{\color{incolor}10}]:} \PY{n}{draw\PYZus{}training\PYZus{}set}\PY{p}{(}\PY{n}{actual}\PY{p}{,}\PY{n}{im\PYZus{}set2}\PY{p}{[}\PY{n}{image\PYZus{}choice}\PY{p}{]}\PY{p}{)}
\end{Verbatim}


    \begin{center}
    \adjustimage{max size={0.9\linewidth}{0.9\paperheight}}{output_17_0.png}
    \end{center}
    { \hspace*{\fill} \\}
    
    \begin{Verbatim}[commandchars=\\\{\}]
Class 3 Training Set
[3319 3530 3600 4297 3981 3148 4789]
3809 3600

    \end{Verbatim}

    \subsection{Problem 2 (Programing): k-Nearest Neighbors Recognition (10
points)}\label{problem-2-programing-k-nearest-neighbors-recognition-10-points}

Instead of using a single nearest neighbor, sometimes it is useful to
consider the consensus (majority vote) of k-nearest neighbours.

    \begin{Verbatim}[commandchars=\\\{\}]
{\color{incolor}In [{\color{incolor}11}]:} \PY{k}{def} \PY{n+nf}{nn2}\PY{p}{(}\PY{n}{im}\PY{p}{,} \PY{n}{k}\PY{p}{,} \PY{n}{order} \PY{o}{=} \PY{k+kc}{None}\PY{p}{,} \PY{n}{imgs}\PY{o}{=}\PY{n}{img} \PY{p}{)}\PY{p}{:}
             \PY{n}{d} \PY{o}{=} \PY{n}{nn1\PYZus{}score}\PY{p}{(}\PY{n}{im}\PY{p}{,} \PY{n}{imgs}\PY{p}{,} \PY{n}{order}\PY{p}{)}
             \PY{k}{return} \PY{n}{np}\PY{o}{.}\PY{n}{argsort}\PY{p}{(}\PY{n}{d}\PY{p}{)}\PY{p}{[}\PY{p}{:}\PY{n}{k}\PY{p}{]}
\end{Verbatim}


    \subsubsection{Repeat Part 1.1 using k-nearest neighbor classifier
(k-NN).}\label{repeat-part-1.1-using-k-nearest-neighbor-classifier-k-nn.}

\paragraph{1. (4 points) Performance}\label{points-performance}

    \begin{Verbatim}[commandchars=\\\{\}]
{\color{incolor}In [{\color{incolor}12}]:} \PY{k}{def} \PY{n+nf}{predict\PYZus{}knn}\PY{p}{(}\PY{n}{im}\PY{p}{,} \PY{n}{k}\PY{p}{,} \PY{n}{order}\PY{p}{)}\PY{p}{:}
             \PY{c+c1}{\PYZsh{} already have the results for k = 1 from Part 1.}
             \PY{k}{if} \PY{p}{(}\PY{n}{k}\PY{o}{==}\PY{l+m+mi}{1}\PY{p}{)}\PY{p}{:}
                 \PY{k}{return} \PY{n}{predict\PYZus{}naive}\PY{p}{(}\PY{n}{im}\PY{p}{,} \PY{n}{img}\PY{p}{,} \PY{n}{label}\PY{p}{)}
             \PY{n}{lab} \PY{o}{=} \PY{n}{np}\PY{o}{.}\PY{n}{empty}\PY{p}{(}\PY{n+nb}{len}\PY{p}{(}\PY{n}{im}\PY{p}{)}\PY{p}{)}
             \PY{k}{for} \PY{n}{j} \PY{o+ow}{in} \PY{n+nb}{range}\PY{p}{(}\PY{n+nb}{len}\PY{p}{(}\PY{n}{im}\PY{p}{)}\PY{p}{)}\PY{p}{:} \PY{c+c1}{\PYZsh{} photo in test set}
                 \PY{n}{classes} \PY{o}{=} \PY{n}{np}\PY{o}{.}\PY{n}{zeros}\PY{p}{(}\PY{l+m+mi}{11}\PY{p}{)}
                 \PY{n}{d\PYZus{}arg} \PY{o}{=} \PY{n}{nn2}\PY{p}{(}\PY{n}{im}\PY{p}{[}\PY{n}{j}\PY{p}{]}\PY{p}{,}\PY{n}{k}\PY{p}{,} \PY{n}{order}\PY{p}{)}
                 \PY{k}{for} \PY{n}{n} \PY{o+ow}{in} \PY{n+nb}{range}\PY{p}{(}\PY{n}{k}\PY{p}{)}\PY{p}{:}
                     \PY{n}{l} \PY{o}{=} \PY{n}{label}\PY{p}{[}\PY{n}{d\PYZus{}arg}\PY{p}{[}\PY{n}{n}\PY{p}{]}\PY{p}{]}
                     \PY{n}{classes}\PY{p}{[}\PY{n}{l}\PY{p}{]} \PY{o}{=} \PY{n}{classes}\PY{p}{[}\PY{n}{l}\PY{p}{]} \PY{o}{+} \PY{l+m+mi}{1}
                 \PY{n}{lab}\PY{p}{[}\PY{n}{j}\PY{p}{]} \PY{o}{=} \PY{n}{np}\PY{o}{.}\PY{n}{argmax}\PY{p}{(}\PY{n}{classes}\PY{p}{)}
             \PY{k}{return} \PY{n}{lab}
\end{Verbatim}


    \begin{Verbatim}[commandchars=\\\{\}]
{\color{incolor}In [{\color{incolor}13}]:} \PY{c+c1}{\PYZsh{}Test the performance for each test dataset using k ∈ 1, 3, 5. }
         \PY{k}{def} \PY{n+nf}{report\PYZus{}knn}\PY{p}{(}\PY{n}{order}\PY{o}{=}\PY{k+kc}{None}\PY{p}{)}\PY{p}{:}
             \PY{k}{for} \PY{n}{k} \PY{o+ow}{in} \PY{n+nb}{range}\PY{p}{(}\PY{l+m+mi}{3}\PY{p}{)}\PY{p}{:}
                 \PY{n+nb}{print}\PY{p}{(}\PY{n}{k}\PY{o}{*}\PY{l+m+mi}{2}\PY{o}{+}\PY{l+m+mi}{1}\PY{p}{)}
                 \PY{k}{for} \PY{n}{i} \PY{o+ow}{in} \PY{n+nb}{range}\PY{p}{(}\PY{l+m+mi}{1}\PY{p}{,}\PY{l+m+mi}{5}\PY{p}{)}\PY{p}{:} \PY{c+c1}{\PYZsh{} test set}
                     \PY{n}{im}\PY{p}{,} \PY{n}{lb}\PY{p}{,} \PY{n}{\PYZus{}} \PY{o}{=} \PY{n}{load\PYZus{}subset}\PY{p}{(}\PY{p}{[}\PY{n}{i}\PY{p}{]}\PY{p}{)}
                     \PY{n}{labels} \PY{o}{=} \PY{n}{predict\PYZus{}knn}\PY{p}{(}\PY{n}{im}\PY{p}{,}\PY{n}{k}\PY{o}{*}\PY{l+m+mi}{2}\PY{o}{+}\PY{l+m+mi}{1}\PY{p}{,}\PY{n}{order}\PY{p}{)}
                     \PY{n}{evaluate}\PY{p}{(}\PY{n}{labels}\PY{p}{,} \PY{n}{lb}\PY{p}{,} \PY{n}{i}\PY{p}{)}
         
         \PY{n}{report\PYZus{}knn}\PY{p}{(}\PY{p}{)}
\end{Verbatim}


    \begin{Verbatim}[commandchars=\\\{\}]
1
Subset 1 accuracy: 94.1667\%
Subset 2 accuracy: 51.6667\%
Subset 3 accuracy: 19.2857\%
Subset 4 accuracy: 15.2632\%
3
Subset 1 accuracy: 94.1667\%
Subset 2 accuracy: 47.5\%
Subset 3 accuracy: 17.8571\%
Subset 4 accuracy: 12.1053\%
5
Subset 1 accuracy: 95.0\%
Subset 2 accuracy: 46.6667\%
Subset 3 accuracy: 17.1429\%
Subset 4 accuracy: 11.0526\%

    \end{Verbatim}

    \paragraph{2. (4 points) L1 Norm
Performance}\label{points-l1-norm-performance}

Use l1 norm rather than the l2 norm to calculate distance.

    \begin{Verbatim}[commandchars=\\\{\}]
{\color{incolor}In [{\color{incolor}14}]:} \PY{n}{report\PYZus{}knn}\PY{p}{(}\PY{l+m+mi}{1}\PY{p}{)}
\end{Verbatim}


    \begin{Verbatim}[commandchars=\\\{\}]
1
Subset 1 accuracy: 94.1667\%
Subset 2 accuracy: 51.6667\%
Subset 3 accuracy: 19.2857\%
Subset 4 accuracy: 15.2632\%
3
Subset 1 accuracy: 90.8333\%
Subset 2 accuracy: 48.3333\%
Subset 3 accuracy: 22.8571\%
Subset 4 accuracy: 13.6842\%
5
Subset 1 accuracy: 93.3333\%
Subset 2 accuracy: 46.6667\%
Subset 3 accuracy: 17.8571\%
Subset 4 accuracy: 14.7368\%

    \end{Verbatim}

    \paragraph{3. (2 points) Comparision}\label{points-comparision}

\subparagraph{With increasing k,}\label{with-increasing-k}

the result did not change much. In some cases it performs better but
generally slightly worse. In the testing set in the previous problem. It
shows that there is usually not a significant variation of distance
number comparing one target picture within the different training sets.
The consistancy results in the k-increase not very effective in better
solving this problem, because the point is to remove outliers but that
usually is not the problem at all in our test images, considering the
consistancy in these normalized training sets. \#\#\#\#\# Changing the
distance metric, the results are not observed to have a siginificant
improvement overall. The fourth set result increased a little bit but
overall slightly down. It may be the result of the change of sensitivity
of variation in the two different algorithms since l1 norm is more
resistant to outliers.

    \subsection{Problem 3 (Programing): Recognition Using Eigenfaces (25
points)}\label{problem-3-programing-recognition-using-eigenfaces-25-points}

\subsubsection{(5 points) Find Principle
Components}\label{points-find-principle-components}

Use Principle Component Analysis (PCA) to get eigenvectors.

    \begin{Verbatim}[commandchars=\\\{\}]
{\color{incolor}In [{\color{incolor}15}]:} \PY{c+c1}{\PYZsh{} input matrix N × dtrainset of vectorized images from subset 0, }
         \PY{c+c1}{\PYZsh{} where N = 70 is the number of training images }
         \PY{c+c1}{\PYZsh{} and d = 2500 is the number of pixels in each training image.}
         \PY{n}{trainset} \PY{o}{=} \PY{n}{np}\PY{o}{.}\PY{n}{reshape}\PY{p}{(}\PY{n}{img}\PY{p}{,}\PY{p}{(}\PY{l+m+mi}{70}\PY{p}{,} \PY{l+m+mi}{2500}\PY{p}{)}\PY{p}{)}
         
         \PY{k}{def} \PY{n+nf}{img\PYZus{}mu}\PY{p}{(}\PY{n}{trainset}\PY{p}{)}\PY{p}{:}
             \PY{n}{d\PYZus{}2500x70} \PY{o}{=} \PY{n}{np}\PY{o}{.}\PY{n}{swapaxes}\PY{p}{(}\PY{n}{trainset}\PY{p}{,} \PY{l+m+mi}{0}\PY{p}{,} \PY{l+m+mi}{1}\PY{p}{)}
             \PY{n}{mu\PYZus{}2500x1} \PY{o}{=} \PY{n}{np}\PY{o}{.}\PY{n}{mean}\PY{p}{(}\PY{n}{d\PYZus{}2500x70}\PY{p}{,} \PY{n}{axis}\PY{o}{=}\PY{l+m+mi}{1}\PY{p}{)}\PY{p}{[}\PY{p}{:}\PY{p}{,}\PY{n}{np}\PY{o}{.}\PY{n}{newaxis}\PY{p}{]}
             \PY{k}{return} \PY{n}{mu\PYZus{}2500x1}\PY{p}{,} \PY{n}{d\PYZus{}2500x70}
         
         \PY{c+c1}{\PYZsh{} Return the d × d matrix of eigenvectors W,}
         \PY{c+c1}{\PYZsh{} that could be further sliced to be a k x d matrix,}
         \PY{c+c1}{\PYZsh{} and a d dimensional vector mu encoding the mean of the training images.}
         \PY{k}{def} \PY{n+nf}{PCA\PYZus{}EIG}\PY{p}{(}\PY{n}{trainset}\PY{p}{)}\PY{p}{:}
             \PY{n}{mu\PYZus{}2500x1}\PY{p}{,} \PY{n}{\PYZus{}} \PY{o}{=} \PY{n}{img\PYZus{}mu}\PY{p}{(}\PY{n}{trainset}\PY{p}{)}
             \PY{n}{s} \PY{o}{=} \PY{n}{np}\PY{o}{.}\PY{n}{zeros}\PY{p}{(}\PY{p}{(}\PY{n+nb}{len}\PY{p}{(}\PY{n}{trainset}\PY{p}{[}\PY{l+m+mi}{0}\PY{p}{]}\PY{p}{)}\PY{p}{,}\PY{n+nb}{len}\PY{p}{(}\PY{n}{trainset}\PY{p}{[}\PY{l+m+mi}{0}\PY{p}{]}\PY{p}{)}\PY{p}{)}\PY{p}{)}
             \PY{k}{for} \PY{n}{i} \PY{o+ow}{in} \PY{n+nb}{range}\PY{p}{(}\PY{n+nb}{len}\PY{p}{(}\PY{n}{trainset}\PY{p}{)}\PY{p}{)}\PY{p}{:}
                 \PY{n}{a} \PY{o}{=} \PY{n}{trainset}\PY{p}{[}\PY{n}{i}\PY{p}{]}\PY{p}{[}\PY{p}{:}\PY{p}{,}\PY{n}{np}\PY{o}{.}\PY{n}{newaxis}\PY{p}{]}
                 \PY{n}{b} \PY{o}{=} \PY{p}{(}\PY{n}{a}\PY{o}{\PYZhy{}}\PY{n}{mu\PYZus{}2500x1}\PY{p}{)}\PY{o}{@}\PY{p}{(}\PY{n}{a}\PY{o}{.}\PY{n}{T}\PY{o}{\PYZhy{}}\PY{n}{mu\PYZus{}2500x1}\PY{o}{.}\PY{n}{T}\PY{p}{)}
                 \PY{n}{s} \PY{o}{=} \PY{n}{s} \PY{o}{+} \PY{n}{b}\PY{o}{/}\PY{p}{(}\PY{n+nb}{len}\PY{p}{(}\PY{n}{trainset}\PY{p}{)}\PY{o}{\PYZhy{}}\PY{l+m+mi}{1}\PY{p}{)}
             \PY{n}{ss}\PY{p}{,} \PY{n}{v} \PY{o}{=} \PY{n}{nl}\PY{o}{.}\PY{n}{eigh}\PY{p}{(}\PY{n}{s}\PY{p}{)}
             \PY{k}{return} \PY{n}{np}\PY{o}{.}\PY{n}{fliplr}\PY{p}{(}\PY{n}{v}\PY{p}{)}\PY{p}{,} \PY{n}{mu\PYZus{}2500x1}
         
         \PY{c+c1}{\PYZsh{} alternative implementation}
         \PY{k}{def} \PY{n+nf}{PCA\PYZus{}SVD}\PY{p}{(}\PY{n}{trainset}\PY{p}{)}\PY{p}{:}
             \PY{n}{mu\PYZus{}2500x1}\PY{p}{,} \PY{n}{d\PYZus{}2500x70} \PY{o}{=} \PY{n}{img\PYZus{}mu}\PY{p}{(}\PY{n}{trainset}\PY{p}{)}
             \PY{n}{x\PYZus{}minus\PYZus{}mu\PYZus{}2500x70} \PY{o}{=} \PY{n}{d\PYZus{}2500x70} \PY{o}{\PYZhy{}} \PY{n}{mu\PYZus{}2500x1}
             \PY{n}{u}\PY{p}{,} \PY{n}{\PYZus{}}\PY{p}{,} \PY{n}{\PYZus{}} \PY{o}{=} \PY{n}{nl}\PY{o}{.}\PY{n}{svd}\PY{p}{(}\PY{n}{x\PYZus{}minus\PYZus{}mu\PYZus{}2500x70}\PY{p}{)}
             \PY{k}{return} \PY{n}{u}\PY{p}{,} \PY{n}{mu\PYZus{}2500x1}
         
         \PY{n}{basis}\PY{p}{,} \PY{n}{mu} \PY{o}{=} \PY{n}{PCA\PYZus{}EIG}\PY{p}{(}\PY{n}{trainset}\PY{p}{)} 
\end{Verbatim}


    \subsubsection{(2 points) Visual Representation of
Eigenvetors}\label{points-visual-representation-of-eigenvetors}

    \begin{Verbatim}[commandchars=\\\{\}]
{\color{incolor}In [{\color{incolor}16}]:} \PY{c+c1}{\PYZsh{} Display these 50 x 50 images by appending them together }
         \PY{c+c1}{\PYZsh{} into a 500 × 100 image (a 10 × 2 grid of images).}
         
         \PY{c+c1}{\PYZsh{}\PYZsh{}\PYZsh{}\PYZsh{}\PYZsh{}\PYZsh{}\PYZsh{}\PYZsh{}\PYZsh{}\PYZsh{}\PYZsh{}\PYZsh{}\PYZsh{}\PYZsh{}\PYZsh{}\PYZsh{}\PYZsh{}\PYZsh{}\PYZsh{}\PYZsh{}\PYZsh{}\PYZsh{}\PYZsh{}\PYZsh{}\PYZsh{}\PYZsh{}\PYZsh{}\PYZsh{}\PYZsh{}\PYZsh{}\PYZsh{}\PYZsh{}\PYZsh{}\PYZsh{}\PYZsh{}\PYZsh{}\PYZsh{}\PYZsh{}\PYZsh{}\PYZsh{}\PYZsh{}\PYZsh{}\PYZsh{}\PYZsh{}\PYZsh{}\PYZsh{}\PYZsh{}\PYZsh{}\PYZsh{}\PYZsh{}\PYZsh{}\PYZsh{}\PYZsh{}\PYZsh{}\PYZsh{}\PYZsh{}\PYZsh{}\PYZsh{}\PYZsh{}\PYZsh{}\PYZsh{}\PYZsh{}\PYZsh{}\PYZsh{}\PYZsh{}\PYZsh{}\PYZsh{}\PYZsh{}\PYZsh{}\PYZsh{}\PYZsh{}\PYZsh{}\PYZsh{}\PYZsh{}\PYZsh{}\PYZsh{}\PYZsh{}\PYZsh{}\PYZsh{}\PYZsh{}\PYZsh{}\PYZsh{}\PYZsh{}\PYZsh{}\PYZsh{}\PYZsh{}\PYZsh{}\PYZsh{}\PYZsh{}\PYZsh{}}
         \PY{c+c1}{\PYZsh{} for better visual presentation, I choose to understand the dimension as Width x Height \PYZsh{}}
         \PY{c+c1}{\PYZsh{}\PYZsh{}\PYZsh{}\PYZsh{}\PYZsh{}\PYZsh{}\PYZsh{}\PYZsh{}\PYZsh{}\PYZsh{}\PYZsh{}\PYZsh{}\PYZsh{}\PYZsh{}\PYZsh{}\PYZsh{}\PYZsh{}\PYZsh{}\PYZsh{}\PYZsh{}\PYZsh{}\PYZsh{}\PYZsh{}\PYZsh{}\PYZsh{}\PYZsh{}\PYZsh{}\PYZsh{}\PYZsh{}\PYZsh{}\PYZsh{}\PYZsh{}\PYZsh{}\PYZsh{}\PYZsh{}\PYZsh{}\PYZsh{}\PYZsh{}\PYZsh{}\PYZsh{}\PYZsh{}\PYZsh{}\PYZsh{}\PYZsh{}\PYZsh{}\PYZsh{}\PYZsh{}\PYZsh{}\PYZsh{}\PYZsh{}\PYZsh{}\PYZsh{}\PYZsh{}\PYZsh{}\PYZsh{}\PYZsh{}\PYZsh{}\PYZsh{}\PYZsh{}\PYZsh{}\PYZsh{}\PYZsh{}\PYZsh{}\PYZsh{}\PYZsh{}\PYZsh{}\PYZsh{}\PYZsh{}\PYZsh{}\PYZsh{}\PYZsh{}\PYZsh{}\PYZsh{}\PYZsh{}\PYZsh{}\PYZsh{}\PYZsh{}\PYZsh{}\PYZsh{}\PYZsh{}\PYZsh{}\PYZsh{}\PYZsh{}\PYZsh{}\PYZsh{}\PYZsh{}\PYZsh{}\PYZsh{}\PYZsh{}\PYZsh{}}
         
         \PY{k}{def} \PY{n+nf}{displayImages}\PY{p}{(}\PY{n}{k}\PY{p}{)}\PY{p}{:}
             \PY{n}{plt}\PY{o}{.}\PY{n}{gcf}\PY{p}{(}\PY{p}{)}\PY{o}{.}\PY{n}{set\PYZus{}size\PYZus{}inches}\PY{p}{(}\PY{l+m+mi}{20}\PY{p}{,}\PY{l+m+mi}{20}\PY{p}{)}
             \PY{n}{a} \PY{o}{=} \PY{n}{np}\PY{o}{.}\PY{n}{vstack}\PY{p}{(}\PY{p}{(}\PY{n}{k}\PY{p}{[}\PY{p}{:}\PY{p}{,}\PY{l+m+mi}{0}\PY{p}{]}\PY{o}{.}\PY{n}{reshape}\PY{p}{(}\PY{p}{(}\PY{l+m+mi}{50}\PY{p}{,}\PY{l+m+mi}{50}\PY{p}{)}\PY{p}{)}\PY{p}{,}\PY{n}{k}\PY{p}{[}\PY{p}{:}\PY{p}{,}\PY{l+m+mi}{10}\PY{p}{]}\PY{o}{.}\PY{n}{reshape}\PY{p}{(}\PY{p}{(}\PY{l+m+mi}{50}\PY{p}{,}\PY{l+m+mi}{50}\PY{p}{)}\PY{p}{)}\PY{p}{)}\PY{p}{)}
             \PY{k}{for} \PY{n}{i} \PY{o+ow}{in} \PY{n+nb}{range}\PY{p}{(}\PY{l+m+mi}{1}\PY{p}{,}\PY{l+m+mi}{10}\PY{p}{)}\PY{p}{:}
                 \PY{n}{x} \PY{o}{=} \PY{n}{k}\PY{p}{[}\PY{p}{:}\PY{p}{,}\PY{n}{i}\PY{p}{]}\PY{o}{.}\PY{n}{reshape}\PY{p}{(}\PY{p}{(}\PY{l+m+mi}{50}\PY{p}{,}\PY{l+m+mi}{50}\PY{p}{)}\PY{p}{)}
                 \PY{n}{y} \PY{o}{=} \PY{n}{k}\PY{p}{[}\PY{p}{:}\PY{p}{,}\PY{l+m+mi}{10}\PY{o}{+}\PY{n}{i}\PY{p}{]}\PY{o}{.}\PY{n}{reshape}\PY{p}{(}\PY{p}{(}\PY{l+m+mi}{50}\PY{p}{,}\PY{l+m+mi}{50}\PY{p}{)}\PY{p}{)}
                 \PY{n}{b} \PY{o}{=} \PY{n}{np}\PY{o}{.}\PY{n}{vstack}\PY{p}{(}\PY{p}{(}\PY{n}{x}\PY{p}{,}\PY{n}{y}\PY{p}{)}\PY{p}{)}
                 \PY{n}{a} \PY{o}{=} \PY{n}{np}\PY{o}{.}\PY{n}{hstack}\PY{p}{(}\PY{p}{(}\PY{n}{a}\PY{p}{,}\PY{n}{b}\PY{p}{)}\PY{p}{)}
             \PY{n}{plt}\PY{o}{.}\PY{n}{imshow}\PY{p}{(}\PY{n}{a}\PY{p}{,}\PY{n}{cmap}\PY{o}{=}\PY{l+s+s1}{\PYZsq{}}\PY{l+s+s1}{gray}\PY{l+s+s1}{\PYZsq{}}\PY{p}{)}
         
         \PY{n}{displayImages}\PY{p}{(}\PY{n}{basis}\PY{p}{)}
\end{Verbatim}


    \begin{center}
    \adjustimage{max size={0.9\linewidth}{0.9\paperheight}}{output_29_0.png}
    \end{center}
    { \hspace*{\fill} \\}
    
    \subsubsection{(2 points) Objective}\label{points-objective}

Performing PCA on the training image observations of \(b\) pixels can
project the observations in \(\Bbb{C}^b\) to a set of values of linearly
uncorrelated feature points (principal components) in \(\Bbb{C}^k\). It
reduces the the amount of data and separtes distinct features of the
observations.

    \subsubsection{(2 points) Reconstruction}\label{points-reconstruction}

Show what each of these images would look like when using only the top k
eigenvectors to reconstruct them, for k = 1, 2, 3, 4, 5, etc.

    \begin{Verbatim}[commandchars=\\\{\}]
{\color{incolor}In [{\color{incolor}17}]:} \PY{n}{p} \PY{o}{=} \PY{p}{\PYZob{}}\PY{p}{\PYZcb{}}
         \PY{c+c1}{\PYZsh{} Select one image per person from subset 0 }
         \PY{c+c1}{\PYZsh{} (e.g., the 5 images person01 01.png, person02 01.png, ... , person10 01.png).}
         \PY{k}{for} \PY{n}{i} \PY{o+ow}{in} \PY{n+nb}{range}\PY{p}{(}\PY{n+nb}{len}\PY{p}{(}\PY{n}{label}\PY{p}{)}\PY{p}{)}\PY{p}{:}
             \PY{k}{if} \PY{p}{(}\PY{n}{seq}\PY{p}{[}\PY{n}{i}\PY{p}{]} \PY{o}{==} \PY{l+m+mi}{1}\PY{p}{)}\PY{p}{:}
                 \PY{n}{p}\PY{p}{[}\PY{n}{label}\PY{p}{[}\PY{n}{i}\PY{p}{]}\PY{p}{]} \PY{o}{=} \PY{n}{img}\PY{p}{[}\PY{n}{i}\PY{p}{]}
         \PY{n}{draw\PYZus{}faces}\PY{p}{(}\PY{n+nb}{list}\PY{p}{(}\PY{n}{p}\PY{o}{.}\PY{n}{values}\PY{p}{(}\PY{p}{)}\PY{p}{)}\PY{p}{)}
\end{Verbatim}


    \begin{center}
    \adjustimage{max size={0.9\linewidth}{0.9\paperheight}}{output_32_0.png}
    \end{center}
    { \hspace*{\fill} \\}
    
    \begin{Verbatim}[commandchars=\\\{\}]
{\color{incolor}In [{\color{incolor}18}]:} \PY{k}{def} \PY{n+nf}{project}\PY{p}{(}\PY{n}{signal}\PY{p}{,} \PY{n}{basis}\PY{p}{,} \PY{n}{mu}\PY{p}{,} \PY{n}{start} \PY{o}{=} \PY{l+m+mi}{0}\PY{p}{,} \PY{n}{k} \PY{o}{=} \PY{l+m+mi}{1}\PY{p}{)}\PY{p}{:}
             \PY{k}{if} \PY{p}{(}\PY{n+nb}{len}\PY{p}{(}\PY{n}{signal}\PY{p}{)}\PY{o}{!=}\PY{l+m+mi}{1}\PY{p}{)}\PY{p}{:}
                 \PY{n+nb}{print}\PY{p}{(}\PY{l+s+s2}{\PYZdq{}}\PY{l+s+s2}{Project Function Error: Signal to decompose needs to be 1 x d dimension.}\PY{l+s+s2}{\PYZdq{}}\PY{p}{)}
                 \PY{k}{return}\PY{p}{;}
             \PY{k}{if} \PY{p}{(}\PY{n+nb}{len}\PY{p}{(}\PY{n}{basis}\PY{p}{)}\PY{o}{!=}\PY{n+nb}{len}\PY{p}{(}\PY{n}{signal}\PY{p}{[}\PY{l+m+mi}{0}\PY{p}{]}\PY{p}{)}\PY{p}{)}\PY{p}{:}
                 \PY{n+nb}{print}\PY{p}{(}\PY{l+s+s2}{\PYZdq{}}\PY{l+s+s2}{Project Function Error: Basis should be of d x k dimension}\PY{l+s+s2}{\PYZdq{}}\PY{p}{)}
                 \PY{k}{return}\PY{p}{;}
             \PY{n}{a} \PY{o}{=} \PY{n}{signal} \PY{o}{\PYZhy{}} \PY{n}{mu}\PY{o}{.}\PY{n}{reshape}\PY{p}{(}\PY{p}{(}\PY{l+m+mi}{1}\PY{p}{,}\PY{o}{\PYZhy{}}\PY{l+m+mi}{1}\PY{p}{)}\PY{p}{)} \PY{c+c1}{\PYZsh{} 1 x d}
             \PY{c+c1}{\PYZsh{} choose first k basis}
             \PY{n}{v} \PY{o}{=} \PY{n}{basis}\PY{p}{[}\PY{p}{:}\PY{p}{,} \PY{n}{start} \PY{p}{:} \PY{n}{start} \PY{o}{+} \PY{n}{k}\PY{p}{]} \PY{c+c1}{\PYZsh{} d x k}
             \PY{c+c1}{\PYZsh{} Project each image into a k dimensional space}
             \PY{n}{b} \PY{o}{=} \PY{n}{a} \PY{o}{@} \PY{n}{v}                       \PY{c+c1}{\PYZsh{} 1 x k }
             \PY{c+c1}{\PYZsh{} project that k dimensional space back into a 2500 dimensional space}
             \PY{n}{c} \PY{o}{=} \PY{n}{v} \PY{o}{@} \PY{n}{b}\PY{o}{.}\PY{n}{T}                     \PY{c+c1}{\PYZsh{} d x 1}
             \PY{c+c1}{\PYZsh{} resize that 2500 vector into a 50 × 50 image.}
             \PY{n}{r} \PY{o}{=} \PY{n}{c}\PY{o}{.}\PY{n}{reshape}\PY{p}{(}\PY{p}{(}\PY{l+m+mi}{50}\PY{p}{,}\PY{l+m+mi}{50}\PY{p}{)}\PY{p}{)} \PY{o}{+} \PY{n}{mu}\PY{o}{.}\PY{n}{reshape}\PY{p}{(}\PY{p}{(}\PY{l+m+mi}{50}\PY{p}{,}\PY{l+m+mi}{50}\PY{p}{)}\PY{p}{)} \PY{c+c1}{\PYZsh{}normalize}
             \PY{k}{return} \PY{n}{b}\PY{o}{.}\PY{n}{T}\PY{p}{,} \PY{n}{c}\PY{p}{,} \PY{n}{r}
             \PY{c+c1}{\PYZsh{} k x 1, d x 1, 50 x 50}
             
         \PY{k}{def} \PY{n+nf}{demon\PYZus{}reconstruction}\PY{p}{(}\PY{n}{start}\PY{o}{=}\PY{l+m+mi}{0}\PY{p}{,} \PY{n}{basis} \PY{o}{=} \PY{n}{basis}\PY{p}{,} \PY{n}{mu} \PY{o}{=} \PY{n}{mu}\PY{p}{)}\PY{p}{:} 
             \PY{n}{plt}\PY{o}{.}\PY{n}{gcf}\PY{p}{(}\PY{p}{)}\PY{o}{.}\PY{n}{set\PYZus{}size\PYZus{}inches}\PY{p}{(}\PY{l+m+mi}{20}\PY{p}{,}\PY{l+m+mi}{6}\PY{p}{)}
             \PY{n}{toDraw}\PY{o}{=}\PY{p}{[}\PY{p}{]}
             \PY{k}{for} \PY{n}{i} \PY{o+ow}{in} \PY{n+nb}{range}\PY{p}{(}\PY{l+m+mi}{1}\PY{p}{,}\PY{l+m+mi}{11}\PY{p}{)}\PY{p}{:} \PY{c+c1}{\PYZsh{} 10 persons}
                 \PY{n}{signal} \PY{o}{=} \PY{n}{p}\PY{p}{[}\PY{n}{i}\PY{p}{]}\PY{o}{.}\PY{n}{ravel}\PY{p}{(}\PY{p}{)}\PY{p}{[}\PY{n}{np}\PY{o}{.}\PY{n}{newaxis}\PY{p}{,}\PY{p}{:}\PY{p}{]}\PY{c+c1}{\PYZsh{} image 1 x 2500}
                 \PY{k}{for} \PY{n}{k} \PY{o+ow}{in} \PY{n+nb}{range}\PY{p}{(}\PY{l+m+mi}{1}\PY{p}{,}\PY{l+m+mi}{11}\PY{p}{)}\PY{p}{:} \PY{c+c1}{\PYZsh{} k=1\PYZhy{}10}
                     \PY{n}{\PYZus{}}\PY{p}{,} \PY{n}{\PYZus{}}\PY{p}{,} \PY{n}{r} \PY{o}{=} \PY{n}{project}\PY{p}{(}\PY{n}{signal}\PY{p}{,} \PY{n}{basis}\PY{p}{,} \PY{n}{mu}\PY{p}{,} \PY{n}{start} \PY{o}{=} \PY{n}{start}\PY{p}{,} \PY{n}{k} \PY{o}{=} \PY{n}{k}\PY{p}{)}
                     \PY{n}{toDraw}\PY{o}{.}\PY{n}{append}\PY{p}{(}\PY{n}{r}\PY{p}{)}
             \PY{n}{draw\PYZus{}faces}\PY{p}{(}\PY{n}{toDraw}\PY{p}{)}
         
         \PY{n}{demon\PYZus{}reconstruction}\PY{p}{(}\PY{p}{)}
\end{Verbatim}


    
    \begin{verbatim}
<matplotlib.figure.Figure at 0x1a1e3eb358>
    \end{verbatim}

    
    \begin{center}
    \adjustimage{max size={0.9\linewidth}{0.9\paperheight}}{output_33_1.png}
    \end{center}
    { \hspace*{\fill} \\}
    
    \subsubsection{(10 points) Recognition}\label{points-recognition}

    \begin{Verbatim}[commandchars=\\\{\}]
{\color{incolor}In [{\color{incolor}19}]:} \PY{k}{def} \PY{n+nf}{projects}\PY{p}{(}\PY{n}{s}\PY{p}{,}\PY{n}{W}\PY{p}{,}\PY{n}{mu}\PY{p}{,}\PY{n}{start}\PY{p}{,}\PY{n}{k}\PY{p}{)}\PY{p}{:}
             \PY{n}{result} \PY{o}{=} \PY{p}{[}\PY{p}{]}
             \PY{c+c1}{\PYZsh{} Project images from trainset onto the space spanned by the first k eigenvectors }
             \PY{k}{for} \PY{n}{i} \PY{o+ow}{in} \PY{n+nb}{range}\PY{p}{(}\PY{n+nb}{len}\PY{p}{(}\PY{n}{s}\PY{p}{)}\PY{p}{)}\PY{p}{:}
                 \PY{n}{kx1}\PY{p}{,} \PY{n}{\PYZus{}}\PY{p}{,} \PY{n}{\PYZus{}} \PY{o}{=} \PY{n}{project}\PY{p}{(}\PY{n}{s}\PY{p}{[}\PY{n}{i}\PY{p}{]}\PY{o}{.}\PY{n}{reshape}\PY{p}{(}\PY{p}{(}\PY{l+m+mi}{1}\PY{p}{,}\PY{o}{\PYZhy{}}\PY{l+m+mi}{1}\PY{p}{)}\PY{p}{)}\PY{p}{,}\PY{n}{W}\PY{p}{,}\PY{n}{mu}\PY{p}{,}\PY{n}{start}\PY{o}{=}\PY{n}{start}\PY{p}{,}\PY{n}{k}\PY{o}{=}\PY{n}{k}\PY{p}{)}
                 \PY{n}{result}\PY{o}{.}\PY{n}{append}\PY{p}{(}\PY{n}{kx1}\PY{p}{)}
             \PY{k}{return} \PY{n}{np}\PY{o}{.}\PY{n}{array}\PY{p}{(}\PY{n}{result}\PY{p}{)}
             
         \PY{c+c1}{\PYZsh{} return an M dimensional vector testlabels encoding the predicted class label for each test example. }
         \PY{k}{def} \PY{n+nf}{eigenTest}\PY{p}{(}\PY{n}{trainset}\PY{p}{,}\PY{n}{trainlabels}\PY{p}{,}\PY{n}{testset}\PY{p}{,}\PY{n}{W}\PY{p}{,}\PY{n}{mu}\PY{p}{,}\PY{n}{k}\PY{p}{,}\PY{n}{start}\PY{o}{=}\PY{l+m+mi}{0}\PY{p}{)}\PY{p}{:}
             \PY{c+c1}{\PYZsh{} Project images from trainset onto the space spanned by the first k eigenvectors }
             \PY{n}{reduced\PYZus{}train} \PY{o}{=} \PY{n}{projects}\PY{p}{(}\PY{n}{trainset}\PY{p}{,}\PY{n}{W}\PY{p}{,}\PY{n}{mu}\PY{p}{,}\PY{n}{start}\PY{o}{=}\PY{n}{start}\PY{p}{,}\PY{n}{k}\PY{o}{=}\PY{n}{k}\PY{p}{)}\PY{p}{[}\PY{p}{:}\PY{p}{,}\PY{p}{:}\PY{p}{,}\PY{l+m+mi}{0}\PY{p}{]} \PY{c+c1}{\PYZsh{} 70 x 10}
             \PY{c+c1}{\PYZsh{} Project images from testset onto the space spanned by the first k eigenvectors}
             \PY{n}{reduced\PYZus{}test} \PY{o}{=} \PY{n}{projects}\PY{p}{(}\PY{n}{testset}\PY{p}{,}\PY{n}{W}\PY{p}{,}\PY{n}{mu}\PY{p}{,}\PY{n}{start}\PY{o}{=}\PY{n}{start}\PY{p}{,}\PY{n}{k}\PY{o}{=}\PY{n}{k}\PY{p}{)}\PY{p}{[}\PY{p}{:}\PY{p}{,}\PY{p}{:}\PY{p}{,}\PY{l+m+mi}{0}\PY{p}{]} \PY{c+c1}{\PYZsh{} 70 x 10}
             \PY{k}{return} \PY{n}{np}\PY{o}{.}\PY{n}{array}\PY{p}{(}\PY{n}{predict\PYZus{}naive}\PY{p}{(}\PY{n}{reduced\PYZus{}test}\PY{p}{,} \PY{n}{reduced\PYZus{}train}\PY{p}{,} \PY{n}{label}\PY{p}{,} \PY{n}{order} \PY{o}{=} \PY{k+kc}{None}\PY{p}{)}\PY{p}{)}
\end{Verbatim}


    \paragraph{Performance}\label{performance}

    \begin{Verbatim}[commandchars=\\\{\}]
{\color{incolor}In [{\color{incolor}20}]:} \PY{k}{def} \PY{n+nf}{testMissRate}\PY{p}{(}\PY{n}{start} \PY{o}{=} \PY{l+m+mi}{0}\PY{p}{)}\PY{p}{:}
             \PY{n}{miss\PYZus{}rate} \PY{o}{=} \PY{n}{np}\PY{o}{.}\PY{n}{empty}\PY{p}{(}\PY{p}{(}\PY{l+m+mi}{4}\PY{p}{,}\PY{l+m+mi}{20}\PY{p}{)}\PY{p}{)}
             \PY{k}{for} \PY{n}{i} \PY{o+ow}{in} \PY{n+nb}{range}\PY{p}{(}\PY{l+m+mi}{4}\PY{p}{)}\PY{p}{:}
                 \PY{k}{for} \PY{n}{k} \PY{o+ow}{in} \PY{n+nb}{range}\PY{p}{(}\PY{l+m+mi}{20}\PY{p}{)}\PY{p}{:}
                     \PY{n}{im}\PY{p}{,} \PY{n}{lb}\PY{p}{,} \PY{n}{\PYZus{}} \PY{o}{=} \PY{n}{load\PYZus{}subset}\PY{p}{(}\PY{p}{[}\PY{n}{i}\PY{o}{+}\PY{l+m+mi}{1}\PY{p}{]}\PY{p}{)}
                     \PY{n}{test\PYZus{}result} \PY{o}{=} \PY{n}{eigenTest}\PY{p}{(}\PY{n}{img}\PY{p}{,}\PY{n}{label}\PY{p}{,}\PY{n}{im}\PY{p}{,}\PY{n}{basis}\PY{p}{,}\PY{n}{mu}\PY{p}{,}\PY{n}{k}\PY{o}{+}\PY{l+m+mi}{1}\PY{p}{,}\PY{n}{start}\PY{p}{)}
                     \PY{c+c1}{\PYZsh{} fraction of incorrect predicted class labels}
                     \PY{n}{miss\PYZus{}rate}\PY{p}{[}\PY{n}{i}\PY{p}{,}\PY{n}{k}\PY{p}{]} \PY{o}{=} \PY{l+m+mi}{1} \PY{o}{\PYZhy{}} \PY{n}{np}\PY{o}{.}\PY{n}{count\PYZus{}nonzero}\PY{p}{(}\PY{p}{(}\PY{n}{np}\PY{o}{.}\PY{n}{array}\PY{p}{(}\PY{n}{test\PYZus{}result}\PY{p}{)}\PY{o}{\PYZhy{}}\PY{n}{lb}\PY{p}{)} \PY{o}{==} \PY{l+m+mi}{0}\PY{p}{)} \PY{o}{/} \PY{n+nb}{len}\PY{p}{(}\PY{n}{lb}\PY{p}{)}
             \PY{k}{return} \PY{n}{miss\PYZus{}rate}
\end{Verbatim}


    \begin{Verbatim}[commandchars=\\\{\}]
{\color{incolor}In [{\color{incolor}21}]:} \PY{n}{miss\PYZus{}rate} \PY{o}{=} \PY{n}{testMissRate}\PY{p}{(}\PY{l+m+mi}{0}\PY{p}{)}
\end{Verbatim}


    \begin{Verbatim}[commandchars=\\\{\}]
{\color{incolor}In [{\color{incolor}22}]:} \PY{n}{plt}\PY{o}{.}\PY{n}{gcf}\PY{p}{(}\PY{p}{)}\PY{o}{.}\PY{n}{set\PYZus{}size\PYZus{}inches}\PY{p}{(}\PY{l+m+mi}{8}\PY{p}{,}\PY{l+m+mi}{3}\PY{p}{)}
         \PY{c+c1}{\PYZsh{} Evaluate eigenTest on each test subset 1\PYZhy{}4 separately}
         \PY{c+c1}{\PYZsh{} for values k = 1...20 (so it should be evaluated 4 × 20 times).}
         \PY{k}{for} \PY{n}{i} \PY{o+ow}{in} \PY{n+nb}{range}\PY{p}{(}\PY{l+m+mi}{4}\PY{p}{)}\PY{p}{:}
             \PY{c+c1}{\PYZsh{} Plot the error rate  of each subset as a function of k in the same plot }
             \PY{n}{plt}\PY{o}{.}\PY{n}{plot}\PY{p}{(}\PY{n}{np}\PY{o}{.}\PY{n}{arange}\PY{p}{(}\PY{l+m+mi}{20}\PY{p}{)}\PY{o}{+}\PY{l+m+mi}{1}\PY{p}{,}\PY{n}{miss\PYZus{}rate}\PY{p}{[}\PY{n}{i}\PY{p}{,}\PY{p}{:}\PY{p}{]}\PY{p}{,}\PY{n}{label}\PY{o}{=}\PY{l+s+s2}{\PYZdq{}}\PY{l+s+s2}{set }\PY{l+s+s2}{\PYZdq{}}\PY{o}{+}\PY{n+nb}{str}\PY{p}{(}\PY{n}{i}\PY{o}{+}\PY{l+m+mi}{1}\PY{p}{)}\PY{p}{)}
             \PY{c+c1}{\PYZsh{} use the Python legend function to add a legend to the plot}
             \PY{n}{plt}\PY{o}{.}\PY{n}{legend}\PY{p}{(}\PY{p}{)}
\end{Verbatim}


    \begin{center}
    \adjustimage{max size={0.9\linewidth}{0.9\paperheight}}{output_39_0.png}
    \end{center}
    { \hspace*{\fill} \\}
    
    \subsubsection{(2 points) Selection of
Eigenvectors}\label{points-selection-of-eigenvectors}

\paragraph{Repeat the experiment from the previous step, but throw out
the first 4
eigenvectors.}\label{repeat-the-experiment-from-the-previous-step-but-throw-out-the-first-4-eigenvectors.}

That is, use k eigenvectors starting with the 5th eigenvector.

    \begin{Verbatim}[commandchars=\\\{\}]
{\color{incolor}In [{\color{incolor}23}]:} \PY{n}{miss\PYZus{}rate\PYZus{}d} \PY{o}{=} \PY{n}{testMissRate}\PY{p}{(}\PY{l+m+mi}{4}\PY{p}{)}
\end{Verbatim}


    Produce a plot similar to the one in the previous step.

    \begin{Verbatim}[commandchars=\\\{\}]
{\color{incolor}In [{\color{incolor}24}]:} \PY{n}{plt}\PY{o}{.}\PY{n}{gcf}\PY{p}{(}\PY{p}{)}\PY{o}{.}\PY{n}{set\PYZus{}size\PYZus{}inches}\PY{p}{(}\PY{l+m+mi}{8}\PY{p}{,}\PY{l+m+mi}{3}\PY{p}{)}
         \PY{k}{for} \PY{n}{i} \PY{o+ow}{in} \PY{n+nb}{range}\PY{p}{(}\PY{l+m+mi}{4}\PY{p}{)}\PY{p}{:}
             \PY{n}{plt}\PY{o}{.}\PY{n}{plot}\PY{p}{(}\PY{n}{np}\PY{o}{.}\PY{n}{arange}\PY{p}{(}\PY{l+m+mi}{20}\PY{p}{)}\PY{o}{+}\PY{l+m+mi}{1}\PY{p}{,}\PY{n}{miss\PYZus{}rate\PYZus{}d}\PY{p}{[}\PY{n}{i}\PY{p}{,}\PY{p}{:}\PY{p}{]}\PY{p}{,}\PY{n}{label}\PY{o}{=}\PY{l+s+s2}{\PYZdq{}}\PY{l+s+s2}{set }\PY{l+s+s2}{\PYZdq{}}\PY{o}{+}\PY{n+nb}{str}\PY{p}{(}\PY{n}{i}\PY{o}{+}\PY{l+m+mi}{1}\PY{p}{)}\PY{p}{)}
             \PY{n}{plt}\PY{o}{.}\PY{n}{legend}\PY{p}{(}\PY{p}{)}
         \PY{n}{plt}\PY{o}{.}\PY{n}{show}\PY{p}{(}\PY{p}{)}
\end{Verbatim}


    \begin{center}
    \adjustimage{max size={0.9\linewidth}{0.9\paperheight}}{output_43_0.png}
    \end{center}
    { \hspace*{\fill} \\}
    
    \paragraph{Analysis}\label{analysis}

The difference in recognition performance is noticible. Looking at the
eigen-faces, it seems like the first four pictures do not contain as
much distinctively separable feature information comparing to that of
the others. So having them may decrease the feature-based recoginition
accuracy.

    \begin{Verbatim}[commandchars=\\\{\}]
{\color{incolor}In [{\color{incolor}25}]:} \PY{n}{displayImages}\PY{p}{(}\PY{n}{basis}\PY{p}{)}
\end{Verbatim}


    \begin{center}
    \adjustimage{max size={0.9\linewidth}{0.9\paperheight}}{output_45_0.png}
    \end{center}
    { \hspace*{\fill} \\}
    
    \subsubsection{(2 points) Selection of k
value}\label{points-selection-of-k-value}

Looking at the performance wrt k value, we observe the decrease of error
rates consistently reflect the increase of the number of eigenvectors
used. By examining a previous example below, it's observable that the
increase of the number will increase the faithful representation of the
original image, making more features visible. So on the contraray side,
a limited number of k might hide many features, making it hard to
recognize the correct corresponding image. The recognition rate drops as
we move from set 1 to set 4. This is not surprising, by examining the
pictures below. It seems like as the lighting angle changes, the
features are harder to distinguish. Thus a lower recognitino rate, a
higher error rate, is expected.

    \begin{Verbatim}[commandchars=\\\{\}]
{\color{incolor}In [{\color{incolor}26}]:} \PY{n}{toDraw} \PY{o}{=} \PY{p}{[}\PY{p}{]}
         \PY{k}{for} \PY{n}{i} \PY{o+ow}{in} \PY{n+nb}{range}\PY{p}{(}\PY{l+m+mi}{1}\PY{p}{,}\PY{l+m+mi}{5}\PY{p}{)}\PY{p}{:}
             \PY{n}{im}\PY{p}{,} \PY{n}{lb}\PY{p}{,} \PY{n}{\PYZus{}} \PY{o}{=} \PY{n}{load\PYZus{}subset}\PY{p}{(}\PY{p}{[}\PY{n}{i}\PY{p}{]}\PY{p}{)}
             \PY{k}{for} \PY{n}{j} \PY{o+ow}{in} \PY{n+nb}{range}\PY{p}{(}\PY{l+m+mi}{10}\PY{p}{)}\PY{p}{:}
                 \PY{n}{toDraw}\PY{o}{.}\PY{n}{append}\PY{p}{(}\PY{n}{im}\PY{p}{[}\PY{n}{j}\PY{p}{]}\PY{p}{)}
         \PY{n}{draw\PYZus{}faces}\PY{p}{(}\PY{n}{toDraw}\PY{p}{)}
\end{Verbatim}


    \begin{center}
    \adjustimage{max size={0.9\linewidth}{0.9\paperheight}}{output_47_0.png}
    \end{center}
    { \hspace*{\fill} \\}
    
    \subsection{Problem 4 (Programing): Recognition Using Fisherfaces (20
points)}\label{problem-4-programing-recognition-using-fisherfaces-20-points}

    \begin{Verbatim}[commandchars=\\\{\}]
{\color{incolor}In [{\color{incolor}27}]:} \PY{k}{def} \PY{n+nf}{class\PYZus{}mu}\PY{p}{(}\PY{n}{trainset\PYZus{}reduced}\PY{p}{,} \PY{n}{label}\PY{p}{)}\PY{p}{:}
             \PY{n}{mu\PYZus{}class} \PY{o}{=} \PY{p}{\PYZob{}}\PY{p}{\PYZcb{}}
             \PY{k}{for} \PY{n}{k} \PY{o+ow}{in} \PY{n+nb}{range}\PY{p}{(}\PY{l+m+mi}{10}\PY{p}{)}\PY{p}{:}
                 \PY{k}{for} \PY{n}{i} \PY{o+ow}{in} \PY{n+nb}{range}\PY{p}{(}\PY{n+nb}{len}\PY{p}{(}\PY{n}{trainset\PYZus{}reduced}\PY{p}{)}\PY{p}{)}\PY{p}{:}
                     \PY{k}{if} \PY{p}{(}\PY{n}{label}\PY{p}{[}\PY{n}{i}\PY{p}{]}\PY{o}{==}\PY{n}{k}\PY{o}{+}\PY{l+m+mi}{1}\PY{p}{)}\PY{p}{:}
                         \PY{n}{w\PYZus{}img} \PY{o}{=} \PY{n}{trainset\PYZus{}reduced}\PY{p}{[}\PY{n}{i}\PY{p}{]}\PY{o}{/}\PY{l+m+mi}{7}
                         \PY{k}{if} \PY{n}{i} \PY{o+ow}{in} \PY{n}{mu\PYZus{}class}\PY{p}{:}
                             \PY{n}{mu\PYZus{}class}\PY{p}{[}\PY{n}{label}\PY{p}{[}\PY{n}{i}\PY{p}{]}\PY{p}{]} \PY{o}{=} \PY{n}{mu\PYZus{}class}\PY{p}{[}\PY{n}{label}\PY{p}{[}\PY{n}{i}\PY{p}{]}\PY{p}{]} \PY{o}{+} \PY{n}{w\PYZus{}img}
                         \PY{k}{else}\PY{p}{:}
                             \PY{n}{mu\PYZus{}class}\PY{p}{[}\PY{n}{label}\PY{p}{[}\PY{n}{i}\PY{p}{]}\PY{p}{]} \PY{o}{=} \PY{n}{w\PYZus{}img}    
             \PY{k}{return} \PY{n}{mu\PYZus{}class}
         
         \PY{c+c1}{\PYZsh{} INPUT}
         \PY{c+c1}{\PYZsh{} N × d matrix trainset of vectorized images from subset 0, }
         \PY{c+c1}{\PYZsh{} the corresponding class labels trainlabels, }
         \PY{c+c1}{\PYZsh{} and the number of classes c = 10}
         \PY{k}{def} \PY{n+nf}{fisherTrain}\PY{p}{(}\PY{n}{trainset}\PY{p}{,} \PY{n}{trainlabel}\PY{p}{,} \PY{n}{c} \PY{o}{=} \PY{l+m+mi}{10}\PY{p}{)}\PY{p}{:}
             \PY{n}{N} \PY{o}{=} \PY{n+nb}{len}\PY{p}{(}\PY{n}{trainset}\PY{p}{)}
             \PY{c+c1}{\PYZsh{} Compute the mean mu of the training data}
             \PY{c+c1}{\PYZsh{} and use PCA to compute the first N − c principal components. Let this be W\PYZus{}PCA.}
             \PY{n}{W\PYZus{}PCA\PYZus{}full}\PY{p}{,} \PY{n}{mu} \PY{o}{=} \PY{n}{PCA\PYZus{}SVD}\PY{p}{(}\PY{n}{trainset}\PY{p}{)}
             \PY{n}{W\PYZus{}PCA} \PY{o}{=} \PY{n}{W\PYZus{}PCA\PYZus{}full}\PY{p}{[}\PY{p}{:}\PY{p}{,}\PY{p}{:}\PY{p}{(}\PY{n}{N}\PY{o}{\PYZhy{}}\PY{n}{c}\PY{p}{)}\PY{p}{]} \PY{c+c1}{\PYZsh{} 2500 x 60}
             \PY{c+c1}{\PYZsh{} Use W\PYZus{}PCA to project the training data into a space of dimension (N − c)}
             \PY{n}{trainset\PYZus{}reduced} \PY{o}{=} \PY{p}{(}\PY{n}{trainset} \PY{o}{\PYZhy{}} \PY{n}{mu}\PY{p}{[}\PY{p}{:}\PY{p}{,}\PY{l+m+mi}{0}\PY{p}{]}\PY{p}{)} \PY{o}{@} \PY{n}{W\PYZus{}PCA} \PY{c+c1}{\PYZsh{} 70 x 60}
             \PY{c+c1}{\PYZsh{} Compute the between\PYZhy{}class scatter matrix S\PYZus{}B on the (N − c) dimensional space from the previous space.}
             \PY{n}{S\PYZus{}B} \PY{o}{=} \PY{n}{np}\PY{o}{.}\PY{n}{zeros}\PY{p}{(}\PY{p}{(}\PY{n}{N}\PY{o}{\PYZhy{}}\PY{n}{c}\PY{p}{,}\PY{n}{N}\PY{o}{\PYZhy{}}\PY{n}{c}\PY{p}{)}\PY{p}{)} \PY{c+c1}{\PYZsh{} 60 x 60}
             \PY{n}{mu\PYZus{}class} \PY{o}{=} \PY{n}{class\PYZus{}mu}\PY{p}{(}\PY{n}{trainset\PYZus{}reduced}\PY{p}{,} \PY{n}{label}\PY{p}{)}
             \PY{n}{mu\PYZus{}reduced} \PY{o}{=} \PY{n}{np}\PY{o}{.}\PY{n}{mean}\PY{p}{(}\PY{n}{trainset\PYZus{}reduced}\PY{p}{,} \PY{n}{axis} \PY{o}{=} \PY{l+m+mi}{0}\PY{p}{)} \PY{c+c1}{\PYZsh{} 60,}
             \PY{k}{for} \PY{n}{i} \PY{o+ow}{in} \PY{n+nb}{range}\PY{p}{(}\PY{n}{c}\PY{p}{)}\PY{p}{:}
                 \PY{n}{b\PYZus{}c} \PY{o}{=} \PY{p}{(}\PY{n}{mu\PYZus{}class}\PY{p}{[}\PY{n}{i}\PY{o}{+}\PY{l+m+mi}{1}\PY{p}{]}\PY{o}{\PYZhy{}}\PY{n}{mu\PYZus{}reduced}\PY{p}{)}\PY{o}{.}\PY{n}{reshape}\PY{p}{(}\PY{p}{(}\PY{o}{\PYZhy{}}\PY{l+m+mi}{1}\PY{p}{,}\PY{l+m+mi}{1}\PY{p}{)}\PY{p}{)}
                 \PY{n}{S\PYZus{}B} \PY{o}{=} \PY{n}{S\PYZus{}B} \PY{o}{+} \PY{n}{N}\PY{o}{*}\PY{p}{(}\PY{n}{b\PYZus{}c}\PY{n+nd}{@b\PYZus{}c}\PY{o}{.}\PY{n}{T}\PY{p}{)} \PY{c+c1}{\PYZsh{} 60 x 60}
             \PY{c+c1}{\PYZsh{} Compute the within class scatter matrix S\PYZus{}W }
             \PY{c+c1}{\PYZsh{} on the (N − c) dimensional space from the previous space.}
             \PY{n}{S\PYZus{}W} \PY{o}{=} \PY{n}{np}\PY{o}{.}\PY{n}{zeros}\PY{p}{(}\PY{p}{(}\PY{n}{N}\PY{o}{\PYZhy{}}\PY{n}{c}\PY{p}{,}\PY{n}{N}\PY{o}{\PYZhy{}}\PY{n}{c}\PY{p}{)}\PY{p}{)} \PY{c+c1}{\PYZsh{} 60 x 60}
             \PY{k}{for} \PY{n}{j} \PY{o+ow}{in} \PY{n+nb}{range}\PY{p}{(}\PY{l+m+mi}{70}\PY{p}{)}\PY{p}{:}
                 \PY{n}{dif} \PY{o}{=} \PY{p}{(}\PY{n}{trainset\PYZus{}reduced}\PY{p}{[}\PY{n}{j}\PY{p}{]} \PY{o}{\PYZhy{}} \PY{n}{mu\PYZus{}class}\PY{p}{[}\PY{n}{label}\PY{p}{[}\PY{n}{j}\PY{p}{]}\PY{p}{]}\PY{p}{)}\PY{o}{.}\PY{n}{reshape}\PY{p}{(}\PY{p}{(}\PY{o}{\PYZhy{}}\PY{l+m+mi}{1}\PY{p}{,}\PY{l+m+mi}{1}\PY{p}{)}\PY{p}{)}
                 \PY{n}{S\PYZus{}W} \PY{o}{=} \PY{n}{S\PYZus{}W} \PY{o}{+} \PY{n}{dif} \PY{o}{@} \PY{n}{dif}\PY{o}{.}\PY{n}{T}
             \PY{c+c1}{\PYZsh{} Compute W\PYZus{}FLD, by solving the generalized eigenvectors }
             \PY{n}{w}\PY{p}{,} \PY{n}{vr} \PY{o}{=} \PY{n}{sl}\PY{o}{.}\PY{n}{eigh}\PY{p}{(}\PY{n}{S\PYZus{}B}\PY{p}{,} \PY{n}{b} \PY{o}{=} \PY{n}{S\PYZus{}W}\PY{p}{)} \PY{c+c1}{\PYZsh{} 60,}
             \PY{n}{vr} \PY{o}{=} \PY{n}{vr}\PY{o}{/}\PY{n}{w}\PY{o}{.}\PY{n}{reshape}\PY{p}{(}\PY{p}{(}\PY{l+m+mi}{1}\PY{p}{,}\PY{o}{\PYZhy{}}\PY{l+m+mi}{1}\PY{p}{)}\PY{p}{)}
             \PY{n}{args\PYZus{}sorted} \PY{o}{=} \PY{n}{np}\PY{o}{.}\PY{n}{argsort}\PY{p}{(}\PY{n}{np}\PY{o}{.}\PY{n}{real}\PY{p}{(}\PY{n}{w}\PY{p}{)}\PY{p}{)}
             \PY{n}{v} \PY{o}{=} \PY{n}{vr}\PY{p}{[}\PY{p}{:}\PY{p}{,}\PY{n}{args\PYZus{}sorted}\PY{p}{[}\PY{p}{:}\PY{p}{:}\PY{o}{\PYZhy{}}\PY{l+m+mi}{1}\PY{p}{]}\PY{p}{]}
             \PY{c+c1}{\PYZsh{} and use the (c−1) largest generalized eigenvalues }
             \PY{n}{W\PYZus{}fld} \PY{o}{=} \PY{n}{v}\PY{p}{[}\PY{o}{\PYZhy{}}\PY{l+m+mi}{9}\PY{p}{:}\PY{p}{]} \PY{c+c1}{\PYZsh{} 9 x 60}
             \PY{n}{W\PYZus{}pca} \PY{o}{=} \PY{n}{W\PYZus{}PCA}\PY{o}{.}\PY{n}{T} \PY{c+c1}{\PYZsh{} 60 x 2500}
             \PY{c+c1}{\PYZsh{} The fisher bases will be a W = W\PYZus{}FLD W\PYZus{}PCA, where W is (c − 1) × d dimensional,}
             \PY{c+c1}{\PYZsh{} W\PYZus{}FLD is (c − 1) × (N − c) dimensional, and W\PYZus{}PCA is (N − c) × d dimensional.}
             \PY{n}{W} \PY{o}{=} \PY{n}{W\PYZus{}fld}\PY{n+nd}{@W\PYZus{}pca} \PY{c+c1}{\PYZsh{} 9 x 2500}
             \PY{k}{return} \PY{n}{W}
\end{Verbatim}


    (5 points) Rearrange the top 9 Fisher bases into images of size 50 × 50
and stack them into one big 450 × 50 image.

    \begin{Verbatim}[commandchars=\\\{\}]
{\color{incolor}In [{\color{incolor}28}]:} \PY{n}{fisher} \PY{o}{=} \PY{n}{fisherTrain}\PY{p}{(}\PY{n}{trainset}\PY{p}{,} \PY{n}{label}\PY{p}{,} \PY{n}{c} \PY{o}{=} \PY{l+m+mi}{10}\PY{p}{)}
         \PY{n}{fisher}\PY{o}{.}\PY{n}{shape}
\end{Verbatim}


\begin{Verbatim}[commandchars=\\\{\}]
{\color{outcolor}Out[{\color{outcolor}28}]:} (9, 2500)
\end{Verbatim}
            
    \begin{Verbatim}[commandchars=\\\{\}]
{\color{incolor}In [{\color{incolor}29}]:} \PY{k}{def} \PY{n+nf}{displayFisher}\PY{p}{(}\PY{n}{fisher}\PY{p}{)}\PY{p}{:}
             \PY{n}{plt}\PY{o}{.}\PY{n}{gcf}\PY{p}{(}\PY{p}{)}\PY{o}{.}\PY{n}{set\PYZus{}size\PYZus{}inches}\PY{p}{(}\PY{l+m+mi}{20}\PY{p}{,}\PY{l+m+mi}{20}\PY{p}{)}
             \PY{n}{a} \PY{o}{=} \PY{n}{fisher}\PY{p}{[}\PY{l+m+mi}{0}\PY{p}{]}\PY{o}{.}\PY{n}{reshape}\PY{p}{(}\PY{p}{(}\PY{l+m+mi}{50}\PY{p}{,}\PY{l+m+mi}{50}\PY{p}{)}\PY{p}{)}
             \PY{k}{for} \PY{n}{i} \PY{o+ow}{in} \PY{n+nb}{range}\PY{p}{(}\PY{l+m+mi}{1}\PY{p}{,}\PY{l+m+mi}{9}\PY{p}{)}\PY{p}{:}
                 \PY{n}{a} \PY{o}{=} \PY{n}{np}\PY{o}{.}\PY{n}{hstack}\PY{p}{(}\PY{p}{(}\PY{n}{a}\PY{p}{,}\PY{n}{fisher}\PY{p}{[}\PY{n}{i}\PY{p}{]}\PY{o}{.}\PY{n}{reshape}\PY{p}{(}\PY{p}{(}\PY{l+m+mi}{50}\PY{p}{,}\PY{l+m+mi}{50}\PY{p}{)}\PY{p}{)}\PY{p}{)}\PY{p}{)}
             \PY{n}{plt}\PY{o}{.}\PY{n}{imshow}\PY{p}{(}\PY{n}{a}\PY{p}{,}\PY{n}{cmap}\PY{o}{=}\PY{l+s+s1}{\PYZsq{}}\PY{l+s+s1}{gray}\PY{l+s+s1}{\PYZsq{}}\PY{p}{)}
             \PY{n}{plt}\PY{o}{.}\PY{n}{gca}\PY{p}{(}\PY{p}{)}\PY{o}{.}\PY{n}{axis}\PY{p}{(}\PY{l+s+s1}{\PYZsq{}}\PY{l+s+s1}{off}\PY{l+s+s1}{\PYZsq{}}\PY{p}{)}
         
         \PY{n}{displayFisher}\PY{p}{(}\PY{n}{fisher}\PY{p}{)}
\end{Verbatim}


    \begin{center}
    \adjustimage{max size={0.9\linewidth}{0.9\paperheight}}{output_52_0.png}
    \end{center}
    { \hspace*{\fill} \\}
    
    (5 points) Perform recognition on the testset with Fisherfaces using
1-NN, and evaluate results separately for each test subset 1-4 for
values k = 1...9. Plot the error rate of each subset as a function of k
in the same plot, and use the legend function in Python to add a
\emph{legend} to your plot.

    \begin{Verbatim}[commandchars=\\\{\}]
{\color{incolor}In [{\color{incolor}30}]:} \PY{k}{def} \PY{n+nf}{testMissRateFisher}\PY{p}{(}\PY{p}{)}\PY{p}{:}
             \PY{n}{miss\PYZus{}rate} \PY{o}{=} \PY{n}{np}\PY{o}{.}\PY{n}{empty}\PY{p}{(}\PY{p}{(}\PY{l+m+mi}{4}\PY{p}{,}\PY{l+m+mi}{9}\PY{p}{)}\PY{p}{)}
             \PY{k}{for} \PY{n}{i} \PY{o+ow}{in} \PY{n+nb}{range}\PY{p}{(}\PY{l+m+mi}{4}\PY{p}{)}\PY{p}{:}
                 \PY{k}{for} \PY{n}{k} \PY{o+ow}{in} \PY{n+nb}{range}\PY{p}{(}\PY{l+m+mi}{9}\PY{p}{)}\PY{p}{:}
                     \PY{n}{im}\PY{p}{,} \PY{n}{lb}\PY{p}{,} \PY{n}{\PYZus{}} \PY{o}{=} \PY{n}{load\PYZus{}subset}\PY{p}{(}\PY{p}{[}\PY{n}{i}\PY{o}{+}\PY{l+m+mi}{1}\PY{p}{]}\PY{p}{)}
                     \PY{n}{test\PYZus{}result} \PY{o}{=} \PY{n}{eigenTest}\PY{p}{(}\PY{n}{img}\PY{p}{,}\PY{n}{label}\PY{p}{,}\PY{n}{im}\PY{p}{,}\PY{n}{fisher}\PY{o}{.}\PY{n}{T}\PY{p}{,}\PY{n}{mu}\PY{p}{,}\PY{n}{k}\PY{o}{+}\PY{l+m+mi}{1}\PY{p}{,}\PY{l+m+mi}{0}\PY{p}{)}
                     \PY{c+c1}{\PYZsh{} fraction of incorrect predicted class labels}
                     \PY{n}{miss\PYZus{}rate}\PY{p}{[}\PY{n}{i}\PY{p}{,}\PY{n}{k}\PY{p}{]} \PY{o}{=} \PY{l+m+mi}{1} \PY{o}{\PYZhy{}} \PY{n}{np}\PY{o}{.}\PY{n}{count\PYZus{}nonzero}\PY{p}{(}\PY{p}{(}\PY{n}{np}\PY{o}{.}\PY{n}{array}\PY{p}{(}\PY{n}{test\PYZus{}result}\PY{p}{)}\PY{o}{\PYZhy{}}\PY{n}{lb}\PY{p}{)} \PY{o}{==} \PY{l+m+mi}{0}\PY{p}{)} \PY{o}{/} \PY{n+nb}{len}\PY{p}{(}\PY{n}{lb}\PY{p}{)}
             \PY{k}{return} \PY{n}{miss\PYZus{}rate}
         
         \PY{n}{result\PYZus{}fisher} \PY{o}{=} \PY{n}{testMissRateFisher}\PY{p}{(}\PY{p}{)}
\end{Verbatim}


    \begin{Verbatim}[commandchars=\\\{\}]
{\color{incolor}In [{\color{incolor}31}]:} \PY{n}{result\PYZus{}fisher}\PY{o}{.}\PY{n}{shape}
\end{Verbatim}


\begin{Verbatim}[commandchars=\\\{\}]
{\color{outcolor}Out[{\color{outcolor}31}]:} (4, 9)
\end{Verbatim}
            
    \begin{Verbatim}[commandchars=\\\{\}]
{\color{incolor}In [{\color{incolor}32}]:} \PY{n}{plt}\PY{o}{.}\PY{n}{gcf}\PY{p}{(}\PY{p}{)}\PY{o}{.}\PY{n}{set\PYZus{}size\PYZus{}inches}\PY{p}{(}\PY{l+m+mi}{8}\PY{p}{,}\PY{l+m+mi}{3}\PY{p}{)}
         \PY{k}{for} \PY{n}{i} \PY{o+ow}{in} \PY{n+nb}{range}\PY{p}{(}\PY{l+m+mi}{4}\PY{p}{)}\PY{p}{:}
             \PY{n}{plt}\PY{o}{.}\PY{n}{plot}\PY{p}{(}\PY{n}{np}\PY{o}{.}\PY{n}{arange}\PY{p}{(}\PY{l+m+mi}{9}\PY{p}{)}\PY{o}{+}\PY{l+m+mi}{1}\PY{p}{,}\PY{n}{result\PYZus{}fisher}\PY{p}{[}\PY{n}{i}\PY{p}{,}\PY{p}{:}\PY{p}{]}\PY{p}{,}\PY{n}{label}\PY{o}{=}\PY{l+s+s2}{\PYZdq{}}\PY{l+s+s2}{set }\PY{l+s+s2}{\PYZdq{}}\PY{o}{+}\PY{n+nb}{str}\PY{p}{(}\PY{n}{i}\PY{o}{+}\PY{l+m+mi}{1}\PY{p}{)}\PY{p}{)}
             \PY{n}{plt}\PY{o}{.}\PY{n}{legend}\PY{p}{(}\PY{p}{)}
         \PY{n}{plt}\PY{o}{.}\PY{n}{show}\PY{p}{(}\PY{p}{)}
\end{Verbatim}


    \begin{center}
    \adjustimage{max size={0.9\linewidth}{0.9\paperheight}}{output_56_0.png}
    \end{center}
    { \hspace*{\fill} \\}
    
    From the graph, it shows the error rate drops as the k number increase.
The challenging lighting conditions still negatively impact the
recognition rate. Fisher mehod has more promising results comparing to
the Eigenface method, especially when applied to the first two dataset.
The extreme darkness in dataset four still significantly impacted the
recognition.


    % Add a bibliography block to the postdoc
    
    
    
    \end{document}
